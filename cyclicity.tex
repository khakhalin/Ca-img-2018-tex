
\begin{document} % --------------------- Document start ----------------

We also introduced a measure of \textbf{cyclicity}, which quantified the presence and average strength of various directed cycles starting from a node and returning back to the same node, relative to a similar value on a full graph without self-loops, with the same number of nodes (see Methods). The cyclicity of observed graphs (1.4$\pm$1.4 e-4 for younger, 3.8$\pm$4.3 e-5 for older tadpoles) was higher than expected for this degree distribution ($d$=0.8 and 0.8, paired $p_t$=0.002 and 0.01 for younger and older animals respectively), but the value for a random graph with a matching degree distribution was lower than for a fully randomized graph ($d$=$-$2.2 and $-$2.6, paired $p_t$=3e-5 and 2e-6 respectively). This non-trivial pattern means that compared to a fully random graph, the distribution of degrees did not favor short cycles (as it had fewer nodes with $k>$2), yet actually observed graphs were structurally enriched with short cycles. There was no change in cyclicity in development ($d$=$-$0.1, $p_t$=0.9).

We also introduced a measure of \textbf{cyclicity}, or the prevalence of short cycles in the graph. We considered nodes of the graph one by one, assumed that constant activation flow arrives to each of these nodes, and let the activation spread across the edges of the graph with probability $d \cdot w_{ji}$ (with $d = $0.9), similar to how we did it for Katz centrality. We then let these activation flows converge to a stable solution, and calculated the total flow that reached each node back via all cycles that started from it and ended in it. We then calculated the total of all these cyclical flows across all nodes, and divided it onto a similar value calculated on a full directed graph without loops with the same number of nodes. In practice cyclic flows to each node can of course be calculated together by introducing a flow matrix $\textbf{S}$, and then going through a sequence of iterations: 

$$\textbf{S} = \textbf{W} \cdot \text{max}(\textbf{S} , \textbf{E}) $$

where $\textbf{E}$ is the identity matrix, and max is an elementwise maximum operator.

\bibliographystyle{apalike} % For author-year
\bibliography{refs}

\end{document}