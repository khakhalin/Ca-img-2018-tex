\title{Khakhalin. Graph analysis of collision detection networks}

%\documentclass[twocolumn]{article}
\documentclass{article}

\usepackage[utf8]{inputenc}
\usepackage[top=0.85in,left=1.5in,right=1.5in,footskip=0.75in]{geometry} % For one-column version
%\usepackage[top=0.85in,left=1in,right=1.0in,footskip=0.75in]{geometry} % For two-column version
% marginparwidth=2in

\usepackage{helvet} % Set font to Arial-like
\renewcommand{\familydefault}{\sfdefault} % Force this font

\usepackage[round, sort, numbers, authoryear]{natbib} % Reference manager, round citations
% \usepackage[super]{natbib} % Nature-style citations
% \setcitestyle{citesep={,}} % For nature-style, comma instead of ;

%\usepackage[switch,pagewise]{lineno} % Numbered lines for two columns
\usepackage[pagewise]{lineno} % Numbered lines for one column

%\usepackage{adjustbox} % To change margins around a table. Doesn't work?
\usepackage{tabularx} % To set table column width, if needed
\usepackage{rotating} % To write text sidewise
\usepackage{xcolor}
\renewcommand{\linenumberfont}{\normalfont\bfseries\small\color{lightgray}}
\definecolor{linkcolor}{rgb}{0.2,0.6,0.7} % Neuron-style link color
\usepackage[colorlinks=true,citecolor=linkcolor,urlcolor=blue]{hyperref}%
%\usepackage{url}

% improves typesetting in LaTeX
\usepackage{microtype}
\DisableLigatures[f]{encoding = *, family = * }

% this package is supposed to give access to upright mu symbol via \micro
%\usepackage{siunitx} % didn't work for some reason
\usepackage{amsmath} % to enable \text tag
\usepackage{amssymb} % to enable \leqslant

% text layout
\raggedright
%\textwidth 6in 
%\textheight 9in
\setlength{\parindent}{0em}
\setlength{\parskip}{1em}

% adjust caption style
\usepackage[aboveskip=5mm,labelfont=bf,labelsep=period,singlelinecheck=off]{caption}

% this is required to include graphics
\usepackage{graphicx}

% Multicolumns
\setlength{\columnsep}{1cm}

% Titles
\usepackage{titlesec}
\titlespacing{\section}{0pc}{0.5pc}{0pc}

% Headers and footers
\usepackage{fancyhdr}
\pagestyle{fancy}
\rhead{Khakhalin AS. Graph analysis of collision detection networks. Page \thepage}
\cfoot{} % To kill footer page numbers

\begin{document} % --------------------- Document start ----------------

%TC:ignore 
% The comment above is for textcount to ignore this text.
% It will ignore everything until the endignore pair, and so on.
% We need to get down to 4500 words in the main text.

% \linenumbers % Comment to suppress line numbers

% title goes here:
%\twocolumn[
\begin{flushleft}
{\Large
\textbf\newline{Graph analysis of collision detection networks in the tectum, and its replication in a simple computational model (DRAFT)}
}
\newline
% authors go here:
\\
Arseny S. Khakhalin\textsuperscript{1,*}
\\
\bigskip
{1} Biology Program, Bard College, Annandale-on-Hudson, NY. 

* Correspondence: khakhalin@bard.edu


% TODO

% Fix figures

% Change Fig. to Figure everywhere

% Sort references by year, to make citations correct

\section*{Abstract}
% 206 words now; should be 200
Looming is one of the most salient visual stimuli an animal may encounter, yet mechanisms of looming detection in vertebrates are poorly understood. For some animals, key computations involved in looming detection and collision avoidance are performed by distributed networks, making it hard to tease out underlying mechanisms. In this study, we contrast a computational developmental model of the optic tectum with the analysis of directed connectivity graphs, reconstructed from high-speed calcium imaging recording in \textit{Xenopus} tadpoles. We report differences in degree distributions and modularity of networks reconstructed from younger and older animals, and show that looming-selective cells tended to serve as activation sinks within the network. We compare these results to predictions from a model governed by spike-time-dependent plasticity, homeostatic intrinsic plasticity, synaptic competition, and driven by structured inputs. We show that the model developed robust looming selectivity, and replicated several effects observed in imaging experiments, including changes in modularity and degree distribution. Other changes predicted by the model (hierarchy, efficiency, and spatial distribution of selective cells) were not observed in imaging experiments. By comparing several reduced models, we predict which developmental rules are most critical for the development of looming selectivity in the brain.
%TC:ignore 
\bigskip

\end{flushleft} % Only relevant for two-column documents, but doesn't hurt
%] % End of one-column region

\section*{Introduction}

Few sensory stimuli are as ill boding for an animal as visual looming. A retinal projection that is small at first, but is quickly growing in size, may promise a painful collision, or a meeting with a predator, and so it inherently calls for an action: an avoidance maneuver, freezing, or defensive posturing, such as a blink. Moreover, to be meaningful, looming detection has to be fast. Not surprisingly, it is described in virtually every type of animals that uses vision, from insects to primates \citep{Pereira2016}.

In different animals, looming detection seems to rely on a variety of network solutions \citep{frost2004review}, including dimming detectors in the retina \citep{ishikane2005,munch2009}, opponent motion detection \citep{klapoetke2017looming}, and competitive spike-frequency adaptation \citep{peron2009adaptation,fotowat2011multiplexing}. Even within a single clade of anuran amphibians (frogs), animals seem to employ at least two competing looming detection mechanisms: non-linear detection of retinal oscillations \citep{baranauskas2012}, and rebound of recurrent activity \citep{khakhalin2014,jang2016}. Moreover, it seems that some of these competing mechanisms may lead to different motor responses, as described in insects \citep{card2008tradeoffs,chan2013avoidance}, fish \citep{burgess2007twoescapes,portugues2009behaviors,budick2000repertoire,temizer2015pathway,bhattacharyya2017assessment}, and tadpoles \citep{khakhalin2014}.

It may seem puzzling that a brain would use several unrelated approaches to solve one practical problem, but this arrangement may make sense developmentally. Simple, crude ways of classifying sensory stimuli can be used to train more sophisticated and efficient networks, capable of nuanced analysis of the sensory world at later stages of development \citep{marblestone2016deeplearning}. For looming stimuli in tadpoles, early in development, when collision detection is still poor \citep{dong2009}, animals may use “hardwired” dimming receptors in the retina \citep{baranauskas2012} both to detect collisions, and to “bootstrap” more sophisticated motion-dependent networks in the tectum. Later, tectal networks can serve as a first line of defense, identifying early phases of looming, and helping motor neurons in the hindbrain to perform fine course correction \citep{khakhalin2014,bhattacharyya2017assessment}, while dimming detectors could remain as a backup, mediating more urgent and less coordinated responses \citep{khakhalin2014}. Moreover, every time collision avoidance is not performed perfectly, sensorimotor networks can be refined, based on inputs from the lateral line and mechanosensory detectors in the skin \citep{felch2016, helmbrecht2018topography}.

Traditionally, the golden standard for describing a behavioral mechanism is to demonstrate that neuronal activation is both sufficient and required for this behavior \citep{krakauer2017reductionist}. This reductionist approach works well in systems that are compact and isolated enough, such as acoustic startle detectors \citep{korn2005mauthner}, or central pattern generators in the spinal cord \citep{roberts2010hatchling}. Yet many systems in the brain are distributed and deeply interconnected, making it hard to represent them as a sum of individual "parts", with distinct functions ascribed to each of these parts \citep{gao2015simplicity}. %It is possible to study complex systems statistically, identifying properties that hold “on average”, and differ in functional and dysfunctional networks, but a statistical approach does not necessarily grant insight into the “meaning” of these properties \citep{bassett2018models}. For example, knowing that a disordered brain does not adhere to the same network statistics as a normal brain would not necessarily tell us how to “fix” this network, even if we had a method to rewire neurons in a targeted fashion.

A more promising approach is to study the origins of structured functional connectivity in the brain. Biological neural networks are almost never truly hardwired, but rather dynamically evolve, governed by developmental rules \citep{pietri2017emergence}, and in interplay with patterned activation that propagates through them \citep{gao2015simplicity}. This insight is encouraging, as developmental rules in the brain act at the level of individual cells and their compartments, such as dendrites and synapses, which implies that while the ultimate product of these rules is exceedingly complex, the rules themselves have to be relatively simple \citep{bassett2018models}. It suggests that our best chance to truly “understand” the brain may lie in identifying the space of rules that lead to the development of functional networks \citep{linderman2017constrain}. This general approach proved to be fruitful in recent years, as several complex network phenomena, including high-order receptive fields \citep{bashivan2018neural}, grid cells \citep{banino2018grid}, and decision circuits \citep{haesemeyer2018convergent}, were shown to evolve spontaneously in systems governed by intuitive developmental rules. 

In this paper, we look at connectomic correlates of looming selectivity in the developing optic tectum of \textit{Xenopus} tadpoles. As a model, tadpole tectum is uniquely suitable for studies of sensory integration, as its neurons are excessively plastic \citep{pratt2007intrinsic,busch2019}, strongly interconnected \citep{james2015}, and develop reliable looming selectivity within about a week of development, as tadpoles mature from developmental stage 46 to stage 49 \citep{dong2009,khakhalin2014}. The refinement of tectal connectivity is dominated by spike-time-dependent plasticity (STDP; \citealt{zhang1998stdp,mu2006stdp}), which is known to favor development of synfire chains \citep{fiete2010chains,zheng2014synfire}: groups of neurons linked with sequential connections that selectively respond to certain patterns of temporal activation \citep{clopath2010stdpcoding}. As  tectal neurons spike relatively slowly, with broad spikes and long refractory periods \citep{ciarleglio2015,jang2016,busch2019}, these synfire chains can be expected to have delays of about 10 ms from one neuron to another, compared to $\sim$2 ms in the mammalian cortex. This slower signal propagation brings connections within the tectum to the threshold of direct detection by fast Ca imaging techniques that operate at rates of ~100 frames/s, allowing observations of not just co-firing of neurons within each ensemble or clique \citep{reimann2017,avitan2017spontaneous}, but propagation of signal through these ensembles.

Here we use high-speed Ca imaging data to infer connectivity of small sub-networks within the tectum, describe connectivity patterns in younger and older tadpoles, and compare them to predictions from a simple developmental simulation. We hypothesize that by comparing statistical properties of networks generated by a family of models to networks reconstructed in imaging experiments we should be able to reverse-engineer developmental rules that govern development of looming-sensitive networks in the tectum. We show that our model predicted some, but not all aspects of topology and functionality in looming-sensitive networks, making it hard to draw simple conclusions. Nevertheless, we report several new findings, including some surprising negative results, and as a proof of concept, demonstrate that looming detectors can emerge in developing networks through a simple interplay of structured inputs and plasticity.

\section*{Results}

For all mathematical methods, we provide their names and their interpretation in the main text, but leave definitions and details for the extended Methods section. For statistical analyses, we report p-values without correction, and interpret them according to Fisher, rather than Neyman-Pearson philosophy \citep{greenland2016}. This approach is preferable, as many of our analyses are not fully independent, but ask different questions, and rely on differently formulated null hypotheses. At the interpretation step, we pay more attention to hypotheses supported by several alternative analyses. All code for this paper is available at \url{https://github.com/khakhalin/Ca-Imaging-and-Model-2018}.

% Todo: once we know where the data is hosted, update

We performed Ca imaging in 14 stage 45-46 tadpoles, and 16 stage 48-49 tadpoles, recording responses from 128$\pm$40 tectal cells (between 84 and 229; here and below “$\pm$” after the mean denotes standard deviation). Unless stated otherwise, sample sizes $n=$ 14 and 16 animals for stage 46 and 49 tadpoles apply to all analyses between younger and older animals in this study. To each tadpole, we presented a sequence of three different stimuli, always in the same order: a dark-on-light "Looming" stimulus, followed by a full-field dark "Flash", followed by a spatially "Scrambled" looming stimulus (Fig 1A). Scrambled stimuli were identical to looming, except that the visual field was split into a 7x7 grid of square tiles, and these tiles were randomly rearranged in space. In total we presented 60$\pm$11 stimuli to every animal, which means that a stimulus of every type was presented 20$\pm$4 times. We recorded high speed calcium imaging signals \citep{xu2011,truszkowski2017} from one layer of “deep” principal tectal neurons in the tectum (Fig 1B,C); extracted fluorescence traces (Fig 1D,E), and inferred instantaneous spiking rate for each neuron within every frame (Fig 1F,G).

\begin{figure*}
\includegraphics[width=\linewidth]{fig1.png}
\caption{
Experimental design overview. \textbf{A}. Four representative frames from visual stimuli of each type. \textbf{B}. Schematic of the preparation. \textbf{C}. View of the optic tectum during the calcium imaging recording. \textbf{D}. Matching regions of interest and average responses in each cell. \textbf{E}. Typical fluorescence responses to flash (F), scramble (S) and looming (L) stimuli from three cells in the tectum. \textbf{F}. Spiking estimations for these fluorescence traces. \textbf{G}. Average full-brain responses to stimuli of every type, wich 95\% confidence interval band, for one representative experiment. }
\end{figure*}

\subsection*{Responses and stimulus selectivity}

As reported previously in electrophysiology experiments \citep{khakhalin2014}, responses to flashes were fast, with a sharp peak and some recurrent activation after the peak, while responses to looming stimuli were slower, and followed by strong recurrent activation (Fig 2A). Responses to both looming and scrambled stimuli were highly variable from one animal to another, which may indicate either an inherent variability of network configurations from animal to animal, or different levels of inhibition across preparations. We did not quantify differences in response shapes, and proceeded with the analysis of response amplitudes.

The total output of observed networks tended to be higher in response to looming stimuli than to flashes (on average, 39$\pm$29\% higher for younger; 25$\pm$25\% higher for older tadpoles; $p_{t1}=$ 2e-4 and 1e-3 respectively; Fig. 2B). There was no change in this preference in development ($p_t$=0.15). There was no difference in response amplitude between looming and scrambled stimuli (average difference of $-$0.03$\pm$0.15 and $-$0.01$\pm$0.20; $p_{t1}=$ 0.45 and 0.78 for younger and older animals respectively; no difference in development $p_t=$ 0.80). These results support our prior observation that the total tectal response in tadpoles depends mostly on the dynamics of visual stimuli, rather than on their geometry \citep{khakhalin2014,jang2016}.

\begin{figure*}
\includegraphics[width=\linewidth]{fig2.png}
\caption{
Selectivity analysis. \textbf{A}. \textbf{B}. \textbf{C}. \textbf{D}. \textbf{E}. \textbf{F}. }
\end{figure*}

To quantify stimulus selectivity, for each tectal cell we calculated Cohen’s $d$ effect sizes between cumulative responses to different stimuli. We considered two measures of selectivity: that for "looming over flash" (a type of selectivity that may rely on both stimulus dynamics and its spatial organization), and "looming over scrambled" (that can only rely on spatial properties of stimuli, as they had the same dynamics). Looking at cumulative responses was of course a simplification, as in real life animals respond to stimuli as they are still unrolling \citep{peron2009adaptation,khakhalin2014}, and not after a timed 1s-long presentation. We however had no way to justify any type of dynamic thresholding, and so opted for the simplest possible approach.

On average, tectal cells were selective for looming stimuli (within-brain mean $d$ of 0.67$\pm$0.50 and 0.46$\pm$0.47 in younger and older tadpoles respectively), with no difference between stages in terms of both mean and variance ($p_t=$ 0.3, 0.3). The share of cells that responded to looming stimuli stronger than to flashes also did not change in development (84$\pm$23\%, 77$\pm$21\%; $p_t=$ 0.4). Compared to younger brains, older brains had fewer highly selective cells (Figure 2I). The gap between top-selective (90th percentile) and median selective cells was larger in younger (0.75$\pm$0.26) than in older tadpoles (0.53$\pm$0.27; $p_t=$ 0.03). These results were unexpected, as older tadpoles perform better in collision avoidance tests \citep{dong2009}, and we expected them to develop a subset of looming-selective cells, as described in adult frogs \citep{nakagawa2010otneurons,baranauskas2012}, and other vertebrates \citep{wang1992pigeon,wu2005pigeon,liu2011cat}. Yet in our experiments a subpopulation of strongly selective cells not only did not emerge in older animals, but became less prominent.

We then considered a second, more computationally demanding definition of selectivity: a preference for spatially organized looming stimuli over scrambled stimuli. On average, tectal cells did not have a preference between these two stimuli (average selectivity of $-$0.07$\pm$0.33 in younger tadpoles, $-$0.04$\pm$0.49 in older ones; no change in development $p_t=$ 0.9). The share of cells that responded to looming stronger than to scrambled was at a chance level for both developmental stages (46$\pm$31\%, 48$\pm$37\%, $p_t=$ 0.9), and there was no change in either within-brain variance of this selectivity ($p_t=$ 0.9), or the 90$-$50 percentile asymmetry of values ($p_t=$ 0.8).

We found that selectivity for scrambled stimuli over flashes correlated with selectivity for looming stimuli over flashes in both developmental groups: within-brain $r=$ 0.82$\pm$0.13, $p_{1t}=$3e-12 for younger animals, and 0.75$\pm$0.18, $p_{t1}=$ 3e-11 older ones (Fig 2D); no change in development ($p_t=$ 0.3). On the contrary, the preference for looming over flashes did not correlate with preference for looming over scrambled (Fig 2E; $r=$ 0.03$\pm$0.29, $p_{1t}=$ 0.7 for stage 46; 0.13$\pm$.30, $p_{1t}=$ 0.1 for stage 49). This confirms that the majority of cells in the tectum responded to stimulus dynamics, rather than to its geometry.

Finally, as a holistic way to quantify tectal network selectivity, we looked at our ability to predict stimuli identity from recorded tectal responses \citep{avitan2016limitations}: a measure known as "stimulus encoding". We ran a multivariate logistic regression on one half of the data, linking amplitudes of responses in each cell to the type of stimulus used in each trial. Then we measured the quality of this linkage on the second half of recorded data (Fig. 2F). The quality of prediction was rather low: 59$\pm$12\% for younger, and 62$\pm$13\% for older tadpoles, with no change in development ($p_t=$ 0.6).

To assess the variability of responses from one tectal cell to another, we performed exploratory factor analysis (principal component analysis, followed by promax rotation) of responses within each preparation. This analysis was restricted to looming stimuli. The first and second principal components explained on average 19$\pm$7\% and 4$\pm$1\% of variance in younger tadpoles, and 24$\pm$14\% and 3$\pm$1\% in older tadpoles, and largely encoded different response timing (Fig. 3A). Cells with early responses to looming stimuli tended to group together within the recorded field (Fig. 3B), and their spatial clustering was reliably identified in each of 30 experiments (see Methods). The most natural explanation for this spatial clustering is that a retinotopic map, which is well known to be present in the tectum \citep{ruthazer2004map}, reproduced a looming stimulus projected on the retina. Across cells, the latency of average response correlated with distance from the retinotopy center ($r=$ 0.35 $\pm$ 0.24; correlations individually significant in 25/30 experiments), despite our latency estimations being noisy, especially for low-amplitude cells (see Methods). Curiously, while visual projections to the tectum are known to be actively remodeled in development \citep{sakaguchi1985refinement,ruthazer2004map,munz2014hebbian}, the quality of functional retinotopic map did not differ between younger and older tadpoles, as correlations between early component prominence and cell position did not change in development (Fig. 3E; $r=$ 0.63 $\pm$ 0.21 and 0.57 $\pm$ 0.25 respectively, $p_t= $0.5), which is similar to reports in Zebrafish \citep{avitan2016limitations}.

\begin{figure*}
\includegraphics[width=\linewidth]{fig3.png}
\caption{
Spatial variability. \textbf{A}. \textbf{B}. \textbf{C}. \textbf{D}. \textbf{E}. \textbf{F}. }
\end{figure*}

Knowing the projection of the looming stimulus center within the tectum, we could then check whether looming-selective cells were more likely to be found in the center of the expanding activation area (as it would be expected if collision detectors themselves formed a retinotopic map \citep{frost2004review}, or if they were based on feedback excitation \citep{jang2016}), or at the periphery. We found that selectivity for looming stimuli over flash tended to decrease with distance from the estimated projection center (Fig. 3G) for both stage 45 (average $r=-$0.37$\pm$0.27; individual correlations $p_r<$0.05 in 12/14 experiments), and stage 49 tadpoles (average $r=-$0.09$\pm$0.35; $p_r<$0.05 in 12/16 experiments), indicating that looming-selective cells tended to be located in the center of the emerging spatial response. Similarly, at both developmental stages, selectivity decreased with response latency (stage 46: $p_r<$0.05 in 11/14 animals, average $r=-$0.29$\pm$0.11; stage 49: $p_r<$0.05 in 10/16 animals, average $r=-$0.16$\pm$0.21). Both correlations were weaker in older tadpoles ($p_t=$ 0.02 for distance-selectivity, $p_t=$ 0.03 for latency-selectivity), indicating a more uniform distribution of looming-selective cells within the network.

\subsection*{Variability and ensembles}

We then assessed the \textbf{trial-to trial variability} of tectal responses, to see whether it changed in development, as it was reported for spontaneous activity in the tectum \citep{xu2011}. We performed principal component analysis of response waveforms, and looked at the total number of components that was needed to describe 80\% of variability in the data \citep{avitan2017spontaneous}. We found that this number was similar in stage 46 and 49 tadpoles, with a minor increase in response richness in older animals (insignificant for each stimulus alone, but consistent across stimuli): 51$\pm$14 and 65$\pm$28, $p_t$=0.1 for responses to looming stimuli; 49$\pm$12 and 62$\pm$32, $p_t$=0.2 for flashes; 51$\pm$14 and 64$\pm$26, $p_t$=0.1 for scrambled stimuli.

To better describe network activation variability, we used spectral clustering technique, and looked for \textbf{ensembles} of tectal cells that were active or silent together on a trial-by-trial basis \citep{thompson2016ensembles}. Unlike for recordings of spontaneous activity, we could not easily aggregate activity states into clusters \citep{avitan2017spontaneous}, as activation in our networks was driven by shared sensory inputs. Instead, we subtracted normalized average responses of each cell from its responses in individual trials, and calculated pairwise correlations on the remaining “anomalies” of trial-by-trial activation (see “Methods”). We turned these pairwise correlations into pairwise distances in a multidimensional space, ran a series of spectral clustering partitions \citep{ng2002spectral}, and of all possible partitions, picked the one that maximized spectral modularity \citep{newman2006modularity,gomez2009community} (see Methods for details). We found that the number of ensembles did not differ between younger and older tadpoles (10$\pm$5 in stage 45, 11$\pm$11 in stage 49; $p_t$=0.9; Fig. 3X), but in older tadpoles ensembles were more coordinated with each other,  producing lower values of network modularity (0.14$\pm$0.05 to 0.09$\pm$0.06, $p_t$=0.03; Fig XXX). Tectal ensembles also tended to be spatially localized (Fig. 3XXX): cells that shared an ensemble were on average closer to each other than to cells from different ensembles, for both younger (29$\pm$9\% closer) and older animals (25$\pm$10\% closer; no difference in development $p_t$=0.3).

\subsection*{Network reconstruction}

\begin{figure*}[t]
\includegraphics[width=\linewidth]{fig4.pdf}
\caption{
Spatial variability. \textbf{A}. \textbf{B}. \textbf{C}. \textbf{D}. \textbf{E}. \textbf{F}. }
\end{figure*}

The high speed of video acquisition used in this study (83 frames/s) allowed us to look not just at instantaneous correlations between activation of individual neurons, but at the propagation of signals through the network. In \textit{Xenopus} preparations, it takes about 5 ms for a presynaptic action potential to evoke neurotransmitter release in a typical tectal synapse \citep{khakhalin2012}, and about 10 ms for a typical synaptic current to depolarize postsynaptic neuron above the spiking threshold \citep{ciarleglio2015,busch2019}, which makes the frame-to-frame delay in our video recordings (12 ms) well posed to infer neuronal connections from activity.

To reconstruct network connectivity, we calculated pairwise transfer entropy \citep{gourevitch2007te,stetter2012te} for activity traces of individual cells. Intuitively, for each pair of neurons $i$ and $j$ we quantified the amount of additional information that past activity of neuron $i$ can offer to predict current activity of neuron $j$. This is conceptually similar to calculating a cross-correlation between the activity of neuron $i$ at each frame $t$, and the activity of neuron $j$ at each frame $t+1$, except that transfer entropy has higher power, lower noise, and does not make assumptions about the type of influence neuron $i$ has over neuron $j$ \citep{stetter2012te}.

In our experiments, all tectal neurons received shared inputs from the eye, that recruited them in similar manner in every trial. This complicated connectivity inference, as neurons could spike in a sequence both because they were connected, and because they received innervation from sequentially activated areas of the retina \citep{mehler2018lure}. To compensate for shared inputs, we randomly reshuffled trials for every neuron, calculated average transfer entropy on reshuffled data, and subtracted it from the value of transfer entropy on trial-matched data \citep{gourevitch2007te,wollstadt2014te}. In essence, instead of looking at raw responses, we worked with deviations from average response, and quantified whether these deviations tended to propagate through the network, from one neuron to another. The reshuffling step also allowed us to calculate a $p$-value for each pair of neurons, and quantify how unusual the actually observed value of transfer entropy was for this pair of neurons, compared to a value arising from shared inputs, but no causal connections.

We interpreted transfer entropy values as approximations of weights in a connectivity matrix $\textbf{W}$ (Fig. 4A), with $w_{ji}$ describing the strength of connection from neuron $i$ to neuron $j$ (but see \citealt{mehler2018lure}). We calculated $\textbf{W}$ and corresponding $p$-values independently on looming, flash, and scrambled stimuli, and used these independent estimations to ensure some level of internal replication within each experiment (see Methods). Of all possible edges, only 1.6$\pm$1.4\% were found to be non-zero in all three independent analyses, which was significantly higher than 0.6$\pm$0.09\% expected if edges were assigned at random (paired t-test $p_{tp}=$ 9e-9). Finally, in each experiment, we introduced cut-offs on edge $p$-values, to exclude weak noisy edges from the connectivity graph. As noise levels varied across recordings, we had to adjusted these cut-offs, making the number of non-zero edges in each reconstruction equal to the number of recorded cells \citep{stetter2012te}. The effective $p$-value cut-offs were between 0.001 and 0.007 depending on the experiment (median of 0.004).

The simplest statistical property of a probabilistic network is its \textbf{degree distribution}: the share of nodes with different number of incoming ($k_{in}$) and outgoing ($k_{out}$) connections. We compared rounded degree distributions (see Methods) between networks detected in younger and older tadpoles (Fig. 4B), and found that more developed networks contained fewer unconnected cells ($k_{in}$=0, $p_t<$0.03) and fewer cells with high number of connections ($k_{in}$=5, $k_{out}$=6, $p_t$=0.01 in both cases), but more cells with intermediate number of connections ($k_{in}$=2, $p_t$=0.001, and $k_{out}$=2, $p_t$=0.04). When we approximated degree distributions (excluding $k$=0) with a power law (Fig. 4C), the power constant $\gamma$ was smaller in younger (1.48$\pm$0.19) than in older tadpoles (1.82$\pm$0.25, $p_t$=2e-4), consistent with a sharper, steeper drop from the rate of occurrence of weakly connected to that of highly connected cells. It suggests (Fig. 4D) that older tectal networks had more chains of connected neurons (degree $k$=1) and forks ($k$=2), while younger neurons had more hyperconnected hubs ($k>$5) and unconnected nodes ($k_{in}$=0), which matches the expectation for STDP-driven networks \citep{fiete2010chains}.

An unusual feature of our calcium imaging protocol, compared to most calcium imaging techniques, was that the signal-to-noise ratio varied greatly from one cell to another, depending on how far it was from the focal plane, and how much dye it absorbed while partially exposed to chamber solution during staining. As a result, the share of cells with weak signals, that appeared unconnected to the rest of the network due to low pairwise transfer entropy, varied across experiments, and was dependent on extraneous parameters, such as the physical curvature of the preparation. To ensure that poorly resolved cells do not bias our analysis too strongly, we restricted further network analysis to the largest weakly connected component of each network. There was no difference in the number of weakly connected components detected in younger and older tadpoles (50$\pm$14 and 50$\pm$26 for stages 45 and 49 respectively, $N$=14 and 16; $p_t$=0.9), but in older animals the largest weakly connected component included a higher share of observed cells (50$\pm$6\% and 64$\pm$12\% for younger and older networks respectively; $p_t$=4e-4), matching the change in degree distributions described above.

Another known theoretical prediction for networks dominated by STDP \citep{mu2006stdp,pratt2008recurrent} is that with time neuronal connections would become highly asymmetric. Indeed, if cells $i$ and $j$ are reciprocally connected, every time $j$ spikes after $i$, STDP would increase the weight $w_{ij}$ , but decrease $w_{ji}$ \citep{abbott1996ltpsequence,fiete2010chains}. We found that in our data, \textbf{the share of bidirectional edges} (with both $w_{ij}$ and $w_{ji}>$0) among all detected edges was smaller (0.3$\pm$0.3\%) than expected for random edge assignment in graphs of our size (0.4$\pm$0.1\%, paired $p_t$=0.02), indicating asymmetric information flow in the tectum. Moreover, the share of bidirectional edges decreased in development, from 0.4$\pm$0.3\% in younger animals to 0.2$\pm$0.2\% in older animals ($p_t$=0.03), suggesting that in tadpoles, STDP is shaping emerging network topology at these developmental stages.

We then looked at whether connected cells were more likely to be located spatially closer to each other. We found that the \textbf{average distance between connected cells} was indeed shorter than the average distance on a randomized graph: 18$\pm$10\% shorter for stage 45, and 17$\pm$8\% shorter for stage 49 tadpoles (individually significant with $p_t<$0.05 for 13/14 and 16/16 experiments respectively). Contrary to our expectations, and in contrast to what is known about visual inputs to the tectum \citep{tao2005refinement}, the intra-tectal connectivity did not become more compact in development ($p_t$=0.7). This may suggest that tectal networks rely on far-reaching recurrent connections to integrate visual information across the visual field \citep{baginskas2009recurrent,liu2016jumbo,jang2016}.

\subsection*{Network properties}

For each connectivity graph reconstructed from tadpole tecta, we calculated several standard network properties, and compared them across developmental stages. We also checked whether our results were statistically unusual for a network of this size, by comparing network properties of our graphs to that of matching randomized graphs \citep{ansmann2012surrogate}. To do so, we randomly rewired edges between nodes, while keeping the distribution of edge weights $w_{ji}$, and the number of non-zero edges adjacent to each node (node degree) fixed, as a generalization of degree-preserving rewiring \citep{maslov2002}. 

\begin{figure*}[t]
\includegraphics[width=\linewidth]{fig5.pdf}
\caption{
Spatial variability. \textbf{A}. \textbf{B}. \textbf{C}. \textbf{D}. \textbf{E}. \textbf{F}. }
\end{figure*}

The measure of average "connectedness" in the graph, known as \textbf{network efficiency}, is defined as the average inverse shortest path for any two nodes in the network \citep{latora2001efficiency}. This value is high when there tend to be short paths between any two randomly chosen nodes, and signals can easily propagate within the graph; the value is low when some nodes are located far from each other on a graph (Figure 5A). For a connectivity graph, high edge weights represent strong connections, so network efficiency is calculated on inverse weights $R_{ji} = 1/w_{ji}$ \citep{rubinov2010toolbox}. We found that network efficiency (0.004$\pm$0.002 for stage 46, 0.002$\pm$0.002 for stage 49 tadpoles) was slightly lower than expected for a randomized network with matching degree distribution ($d$=$-$0.3, paired $p_t$=0.04 and $d$=$-$0.3, paired $p_t$=0.06 for younger and older tadpoles respectively). Efficiency did not seem to change with age ($d$=$-$0.8, $p_t$=0.06).

Network \textbf{clustering coefficient} describes the small-scale heterogeneity in the network \citep{fagiolo2007}, and is defined as the relative frequency of two neighboring nodes being a part a triangle with a third node connected to both of them (Figure 5B). The value of clustering coefficient in our networks was very small (2.4$\pm$2.5 e-3 for stage 46, 1.5$\pm$1.6 e-3 for stage 49 animals), but slightly larger than expected in a randomly rewired network with same degree distribution ($d=$ 0.5 and 0.6, paired $p_t=$ 0.01 and 0.02 for younger and older animals). This means that neurons with more than two connections were more likely to form clusters than one would have expected it to happen by chance. There was no change in clustering in development ($d$=$-$0.4, $p_t$=0.3, Fig.).

Network \textbf{modularity} is the most commonly used measure of mesoscale network heterogeneity \citep{leicht2008community,newman2006modularity}. A network with high modularity can be split into a set of sub-networks, with higher density of connections within each sub-network, and weaker connections between sub-networks (Fig. XXX example). In our experiments, reconstructed tectal networks were slightly more modular than randomly rewired networks ($d=$ 0.2 and 0.3, paired $p_t=$ 0.2 and 0.06, for stages 46 and 49 respectively), suggesting a structural non-randomness of network organization. Network modularity increased in development ($d=$ 1.0, $p_t =$ 0.01).

\textbf{Hierarchical flow} is a measure of network hierarchy \citep{mones2012hierarchy} that we based on the distribution of Katz centrality values within the network (\citealt{katz1953original,fletcher2018katz}; see below for definitions). Intuitively, flow hierarchy is high when connections between nodes largely point in the same direction, as it happens in layered feed-forward networks, networks with chains of directed edges, or with activity sinks \citep{czegel2015hierarchy}. We hypothesized that a network of dedicated looming detectors may exhibit flow hierarchy, with edges leading from "feeder neurons" to "detector neurons". Indeed, tectal networks were more hierarchical than randomized networks with matching degrees distributions ($d=$ 1.5 and 1.1, paired $p_t=$ 1e-04 and 1e-03 for younger and older tadpoles respectively). There was no difference in hierarchical flow in development ($d=-$0.3, $p_t=$ 0.4).

\subsection*{Selectivity Mechanisms}

Even though the exact architecture of collision-detecting tectal networks is unknown, it is safe to assume that looming-selective neurons integrate streams of information from different parts of the visual field. We therefore hypothesized that the position of selective neurons within connectivity graphs would not be random \citep{timme2016degree}. To verify that, for each reconstructed network we tested correlations between looming selectivity of each neuron and values that quantify its place within the network, known as measures of \textbf{centrality}.

To identify information sinks, for each cell we calculated its \textbf{Katz centrality} within the graph \citep{katz1953original,fletcher2018katz}. By definition, nodes with high Katz centrality have many paths leading to them, so a spike originating at random within a graph is more likely to eventually cause activation of these nodes, compared to a node with a low Katz centrality. We found that when all cells from all experiments are considered (Figure 6A), there was a weak correlation between looming selectivity of cells and their Katz centrality ($r =$ 0.02, $p_{r} =$ 1e-6, $n=$2487). Correlation coefficients between Katz centrality and selectivity were positive in 19/30 experiments (Figure 6B; average $r=$ 0.09$\pm$0.20 $p_{t1}=$ 0.03), and there was no difference in this effect in development ($p_t=$ 0.5).

A node may have high Katz centrality for two reasons: it may receive a higher number of incoming connections (higher in-degree), or have chains of directed edges lead to it. To see which of these patterns may be at work here, we checked whether looming-selective cells received more incoming connections, compared to non-selective cells \citep{litwin2014assemblies}. When all points were considered, there was no correlation between in-degree and cell selectivity ($p_{r}=$0.3, $n=$ 2487), but for individual experiments correlation coefficients between in-degree and selectivity were positive in 20/30 of cases (Figure 6C, $p_{t1}=$ 0.03). There was no change of this correlation in development ($p_t=$0.5).

% If necessary, we may add back a statement that they didn't get more projections out, in contradiction with [timme2016degree], which is probably because they would be projecting out of the network, and not in?

As cells with higher Katz centrality tend to be activated more often \citep{fletcher2018katz}, we checked whether selectivity for looming stimuli correlated with cell \textbf{average spiking activity} during our recordings. We found that for both younger and older animals, actively spiking cells tended to have stronger looming selectivity (across all cells $r_S=$ 0.07, $p_r=$ 4e-4; average $r_S =$ 0.34$\pm$0.20 and 0.14$\pm$0.26 respectively; in both cases $r$ is significantly $>0$, $p_{t1}=$ 2e-5 and 0.04), which matches predictions from our earlier studies of intrinsic excitability in the tectum \citep{busch2019}. The size of this correlation decreased in development ($p_t=$ 4e-4). (For this analysis we used Spearman rather than Pearson correlation, as 4 experiments had single neurons with extremely high activity levels that acted as influential points. Excluding these 4 neurons and using Pearson correlation leads to similar results.) 

If looming-selective cells gather information from the network, it was plausible to expect that they could be more likely connected to other selective cells, rather than to non-selective ones. To test this, we looked into \textbf{assortativity of selectivity}: a weighted correlation between selectivity scores of cells connected by edges, with strength of these edges used as weight (see Methods). We found that in both younger and older tadpoles selectivities of connected cells correlated (for stage 46: $r$=0.07$\pm$0.13, $p_{t1}$=0.04, individual $r>$0 in 9/14 experiments; for stage 49: $r$=0.07$\pm$0.10, $p_{tq}$=0.02; individual $r>$0 in 10/16 experiments). This suggests that similarly selective cells tended to be connected to each other. There was no change in this effect over development ($p_t=$ 0.9).

% LOOK AT THE PLOT

Finally, we checked whether higher average activity of looming-selective cells linked them into tight clusters, or "rich clubs". We calculated \textbf{local clustering coefficient} for each cell, and checked whether it correlated with cell selectivity. We found that local clustering coefficient did not correlate with cell selectivity when all cells were considered ($p_r=$ 0.6, n=2487), and across all reconstruction, correlation coefficients observed in individual experiments were not different from zero ($p_t1=$ 0.25, n=30), indicating that while selective cells tended to be connected to each other, they did not form clusters.

Together these results suggest that the distribution of selective cells in tectal networks was not random. It is particularly true in younger animals, where selective cells tended to serve as activation sink, and showed higher levels of activation.

We also hypothesized that selectivity for looming stimuli may gradually “accumulate” as the signal propagates through the network, and so “downstream” cells on the receiving end of a strong edge within a graph would, on average, be more selective than “upstream” cells. This hypothesis proved to be wrong in both younger and older tadpoles: across all experiments, 52$\pm$7\% of strong edges (top 25\% of $w_{ji}$ values) led from cells with less to more selective cells, which is not different from chance rate ($p_{t1}$=0.1 for analysis across experiments, $n$=30). A weighted average increase in selectivity between two cells connected by an edge was 0.02$\pm$0.07 (not different from zero: $p_{t1}$=0.2, $n$=30), and there was no change in this value in development ($p_t$=0.2, $n$=14, 16).

\subsection*{Developmental Model}

\begin{figure*}[t!]
\includegraphics[width=\linewidth]{fig6.png}
\caption{
Spatial variability. \textbf{A}. \textbf{B}. \textbf{C}. \textbf{D}. \textbf{E}. \textbf{F}. }
\end{figure*}

As tectal networks are complex and noisy, it was hard to provide a clear mechanistic interpretation of our experimental results. To compensate for this limitation, and to provide a theoretical counterpoint to our experimental observations, we built a mathematical model of the developing tectum, and ran this model through the same set of measurements that we applied to our experiments. The model consisted of 81 artificial neurons, arranged in a 9$\times$9 grid, and all originally connected to each other (every neuron to every neuron) with random positive (excitatory) synaptic weights (Fig XXX). The model operated in 10 ms increments, and we interpreted the output of each neuron as its instantaneous firing rate. At each time step we looked at the network activity in the previous step, and summed up the inputs each neuron would receive from each other neuron. We then used a sliding logistic function to translate these synaptic inputs into postsynaptic spiking (see “Methods” for details).

To allow the network develop in time, we introduced three simple developmental rules: spike-time-dependent plasticity (STDP), homeostatic plasticity, and synaptic competition. Our implementation of STDP approximated biological STDP, as observed in the tadpole tectum \citep{zhang1998stdp,mu2006stdp}: if two cells were active in two consecutive time frames, and they were connected with a synapse, the weight of this synapse was increased. Conversely, if two cells were active within the same time frame, the weight of a synapse connecting them was decreased, as it means that one cell would try to activate another cell after it had already spiked. The homeostatic plasticity rule adjusted excitability thresholds, trying to keep spiking output of each neuron constant on average \citep{pratt2007intrinsic,turrigiano2011}. The rule of synaptic competition attempted to keep constant the total strength of synaptic inputs to each neuron, as well as the total strength of outputs from each neuron, scaling all synapses of both input and output neurons towards a set target every time there was a change in synaptic efficiency \citep{hamodi2016nmda,cohen2002synreview,munz2014hebbian}.

With these three rules at play, we exposed the model to patterned sensory stimulation, modeling retinotopic inputs from the eye. We hypothesized that in biological networks, STDP-driven changes may be amplified by a learning signal \citep{savin2014stdpreward,aswolinskiy2015stdpreward} originating either in dimming receptors in the retina \citep{baranauskas2012}, or mechanosensory systems of the hindbrain \citep{pratt2009multisens,felch2016,truszkowski2017}. We therefore only exposed our model to looming stimuli (but see sensitivity analyses described below). The network was allowed to develop for 12500 time steps (at least 500 looming stimuli), and we saved its topology at five equally spaced time points during this process, from a naive network, to the final state. We ran independent simulations 50 times, and for each network snapshot analyzed the connectivity graph, as well as the responses of each network to “model visual stimuli” that included a looming stimulus, a scrambled stimulus, and a full-field flash. We then analyzed these data in the same way as we did for biological experiments.

\begin{figure*}[t!]
\includegraphics[width=\linewidth]{fig7.png}
\caption{
Model results. \textbf{A}. \textbf{B}. \textbf{C}. \textbf{D}. \textbf{E}. \textbf{F}. }
\end{figure*}

The summary of modeling results is shown in Figure 7. In development, the network became selective for looming stimuli, both in terms of the total response (by the end of development it responded to looming 99$\pm$9\% stronger than to flash) and mean selectivity (mean Cohen $d$ of responses = 1.09$\pm$0.10). The share of cells selective for looming stimuli also increased, and then saturated at $\sim$98\% level, as did the selectivity of the top 10\% of most selective cells.

We then tested whether in our model the nature of visual stimuli could be reconstructed from network activation (stimulus encoding). For a dataset, consisting of equal shares of looming and non-looming stimuli, stimulus encoding increased in development, and plateaued at the prediction accuracy level of $\sim$95\%. This suggests that a retinotopic STDP-driven network can achieve highly reliable looming detection if it is equipped with an output layer, potentially representing motor neurons in the hindbrain \citep{helmbrecht2018topography}, and the weights of connections to the output layer can be tuned, e.g. via reinforcement learning.

Unlike in biological experiments, the model was selective for looming stimuli over scrambled stimuli at the full network level, and by the end of the training period, responses to looming stimulus were 43$\pm$7\% stronger than to scrambled. Mean selectivity of individual cells, defined as a Cohen $d$ for responses to looming compared to scramble, was 0.48$\pm$0.07; and 84\% of cells were selective for looming stimuli, which was also different from what we observed in biological experiments. The selectivity for looming over flash correlated with selectivity for scrambled over flash on a cell by cell basis ($r$= 0.17$\pm$0.10). A subset of highly selective cells did not emerge in the model, and the difference between 90th and 50th percentiles of selectivity were rather low ($\sim$ 0.9 ; Fig 5F).

The positioning of selective cells within the network was different in the model, compared to biological experiments. While in tadpoles, selective cells tended to be located in the middle of the retinotopic field, in the model they tended to be located on the periphery, and selectivity for looming stimuli positively correlated with the distance from the network center. Similar to tadpoles, however, this correlation disappeared in development (from $r\sim$ 0.75 in a naive network, to $r\sim$0.25 in a trained network). Despite the fact that the edges of looming stimuli generally traveled from the center of the retinotopic field to the periphery, there was no correlation between the weight of a cell-to-cell connection and its orientation away from the network center ($r$=8e-5), which means that connections were equally likely to face outward and inward. Similarly to biological networks, selective cells tended to be closer to each other (29$\pm$3\% closer) than expected by chance, yet unlike for biological networks, this locality of connections was refined in developments.

We then looked at topological and functional correlates of looming selectivity in model networks. Selective cells tended to be more spiky ($r$=0.34$\pm$0.12), and usually were not a part of a cluster (final correlation with local clustering coefficient $r$=$-$0.26$\pm$0.12). Unlike in biological networks, in the model selective sells were not special in terms of their Katz centrality ($r$=0.02$\pm$0.13), and they did not tend to receive an unusual number of incoming connections ($r=$0.00$\pm$0.12). As in biological experiments, selective cells tended to be connected to each other (weighted assortativity of 0.24$\pm$0.07). In naive networks, the majority ($\sim$85\%) of strong edges (top 50\% of edges by weight) tended to lead from less selective sells to more selective ones, but in developed networks this share was reduced to chance value (51$\pm$0.03), matching our results from biological networks.

The variability of responses to looming stimuli over time, quantified as the number of principal components required to describe 80\% of response variability, mildly increased with network development, from 28$\pm$1 to 37$\pm$1. The number of ensembles detected in the network did not change, and stayed around 8 to 8.5$\pm$0.5, but the share of variance in network responses explained by the involvement of different ensembles increased from $\sim$35\% in naive networks, to $\sim$50\% by the end of learning. As in biological results, cells that formed an ensemble were about 10\% closer to each other than random two cells in the network, and were 2.2$\pm$0.5 times more likely to be connected to each other.

The distribution of degrees in the model was similar to that in biological experiments: the share of weakly connected cells (weighted in-degree $<$ 0.5) plummeted from $\sim$50\% in naive networks to 9$\pm$2\% by the end of training. On the contrary, the share of cells with degrees of 1 and 2 increased from $\sim$40\% to 83$\pm$2\%. With this changes, the power constant for the degree distribution changed from $\gamma \sim -$1.5 to $-$1.96$\pm$0.03 for both in- and out-degrees, which also qualitatively matched changes in biological experiments.

Finally, we observed that most network measures changed with network development (Figure 6): efficiency and modularity increased, while clustering decreased. In all three cases, the changes were mainly due to changes in weight and degree distribution, as they persisted if calculated on networks randomized with degree-preserving rewiring. The hierarchical flow increased mildly in development, which was entirely due to structured changes in network topology, as the effect was not present in rewired graphs. A robust increase in modularity may seem to be in contradiction with a stable number of ensembles detected in the network, as network modules are expected to form activity ensembles \citep{triplett2018emergence}. We assume that the reason for this difference was that both in biological and computational experiments, we tried to identify ensembles in highly structured responses, and not in long recordings of spontaneous activity.

To conclude whether most predictions of the model were confirmed in experiments, we formulated a list of “atomic”, elementary statements about each of the measures were analyzed, looking at whether they changed in development, whether they increased or decreased, and whether they differed from similar measures in a randomized network (Table 1, first two columns). Overall, the model and the experiments showed similar selectivity for looming, but different selectivity to scrambled stimuli. The interplay between cell position and connectivity was also similar, except for the spatial distribution of looming-selective cells within the retinotopic map, which was peripheral in the model, but central in tadpoles. Changes in degree distributions were well matched, but with a possible exception of modularity, none of other network measures matched. When correlations between different measures of node centrality and cell selectivity were considered, some of them matched, but most did not.

%\newgeometry{left=1in} % This page will have smaller margins
\begin{table}
    % Help for tables:
% https://en.wikibooks.org/wiki/LaTeX/Tables

\newcolumntype{L}{>{$}l<{$}} % math-mode version of "l" column type
\begin{tabular}{lLLLLLLL}
\textbf{Observation} & 
\multicolumn{1}{|l}{} & 
\multicolumn{6}{|c}{\textbf{Model}}\\
\cline{3-8} &
\multicolumn{1}{|l}{\begin{turn}{90}\textbf{Imaging}\end{turn}} & 
\multicolumn{1}{|l}{
  \begin{turn}{90}Base\end{turn}
} & 
%\text{-STDP} & 
\begin{turn}{90}\text{No STDP}\end{turn} &
%\begin{tabular}{l}No\\Syn.\end{tabular} & 
\begin{turn}{90}\text{No Syn. Comp. }\end{turn} &
%\begin{tabular}{l}No\\Intrins.\end{tabular} & 
\begin{turn}{90}\text{No Instrinsic}\end{turn} &
%\text{Visual} & 
\begin{turn}{90}\text{Visual}\end{turn} &
%\text{Noise} 
\begin{turn}{90}\text{Noise}\end{turn}
\\
\hline
Full brain FL selectivity & \checkmark & \checkmark & \checkmark & \checkmark & \times & \checkmark & \checkmark \\
Brain selectivity, change & =	& \land	\lor	& 	\land \lor	& \land & = & \land & =\\
Av. FL selectivity & \text{0.6} & \text{1.0} & \text{1.0} & \text{0.7} & -\text{7} & \text{0.8} & \text{0.5} \\
Av. FL selectivity, change & = & \land & \land \lor & \land & \lor \land & \land & = \\
\% FL select. cells & \text{80\%} & \text{97\%} & \text{99\%} & \text{70\%} & \text{0\%} & \text{95\%} & \text{95\%}\\
\% FL select. cells, change & = & \land & \land \lor & \land \lor & = & \land & \land\\
\hline
Full brain SL selectivity & \times & \checkmark & \checkmark & \checkmark & \times & \checkmark & \times  \\
Av. SL selectivity & -\text{0.1} & \text{0.5} & \text{0.4} & \text{0.5} & -\text{0.4} & \text{0.2} & \text{0}\\
Av. SL selectivity, change & = & \land & \land \lor & \land & \lor & \land & = \\
\% SL select. cells & \text{50\%} & \text{85\%} & \text{85\%} & \text{75\%} & \text{20\%} & \text{70\%} & \text{50\%} \\
\% SL select. cells, change & = & \land & \land \lor & \land & \land \lor & \land & \lor \\
%SL skew, change & = & \land & \land \lor & \land & - & \land \lor & -\\
cor(FS, FL) & \checkmark & \times & \checkmark & \checkmark & \checkmark & \times & \times \\
cor(SL, FL) & \times & \checkmark & \checkmark & \checkmark & \checkmark & \checkmark & \checkmark\\
Stimulus encoding, change & = & \land & \land & \land & \lor & \land & = \\
PCA, N components & = & \land & \land & \land \lor & \land & \land & \land\\
\hline
\multicolumn{8}{l}{\textbf{Ensembles:}}\\
N Ensembles, change & = & = & = & \lor & \land & = & = \\
Spatial locality & \checkmark & \checkmark & \checkmark & \checkmark & \checkmark & \checkmark & \checkmark\\
Preferential connections  & \checkmark & \checkmark & \checkmark & \checkmark & \checkmark & \times & \times \\
\hline
\multicolumn{8}{l}{\textbf{Connected cells:}}\\
Are spatially close & \checkmark & \checkmark & \checkmark & \checkmark & \checkmark & \checkmark & \times \\
%Get closer in development& \times & \times & \times & \times & \times & \times & \times \\
%Edges point to periphery & = & = & = & \checkmark & = & = & =\\
\% two-way edges, change & \lor & \lor & \land & \lor & \lor & \lor & \lor \\
%Edges increase selectivity & \times & \times & \times & \checkmark & \times & \times & \times \\
\hline
\multicolumn{8}{l}{\textbf{Network properties:}}\\
%Degree $k=0$, change & \lor & \lor & \lor & \land & \lor & \lor & \lor \\
%Degree $k=1-2$, change & \land & \land & \land & \lor & \land & \land & \land \\
%Degree $k\geqslant 5$, change & \lor & \lor & \lor & \land & \lor & \lor & \lor \\
Degree power ($\gamma$), change & \land & \land & \land & \lor & \land & \land & \land \\
Efficiency, change & = & \land & = & = & \land & \land & \land \\
Clustering, change & = & \lor & \lor & \lor & \lor & \lor & \lor \\
Modularity, change & \land & \land & \land & \land & \land & \land & \land \\
Hierarchy, change & = & \land & \land \lor & \land & \land & \land & =\\
\hline
\multicolumn{8}{l}{\textbf{Properites of selective cells:}}\\
Center / Periphery location & \text{C} & \text{P} & \text{C} & \text{P} & \text{C} & \text{P} & \text{P}\\
Spatial grouping, change & \lor & \lor & \lor & \lor & \lor & \lor & = \\
High in-degree $k_{in}$ & \checkmark & \times & \times & \checkmark & \times & \times & \times\\
High Katz rank & \checkmark & \times & \times & \checkmark & \times & \times & \times\\
High activity & \checkmark & \checkmark & \times & \checkmark & \checkmark & \checkmark & \times\\
Assortatively connected & \checkmark & \checkmark & \checkmark & \checkmark & \checkmark & \checkmark & \checkmark\\
%Have unusual clustering coeff. & = & \text{low} & = & \text{high} & \text{low} & = & =\\
\end{tabular}
    \caption{A summary of network phenomena observed in biological experiments, in comparison with the base model, and several reduced models. For clarity, we use $\checkmark$ for "yes", $\times$ for "no", $\land$ for "increase", $\lor$ for "decrease", $\land \lor$ for "increase followed by decrease", and $=$ for "no change". FL stands for "Flash-Looming" comparisons; SL - Scrambled-Looming comparisons; corr denotes correlation.}
\end{table}
%\restoregeometry

\subsection*{Sensitivity analysis}

While a comparison with one faithfully constructed model is important, a better approach is to consider a family of models, and see which elements are critical for the replication of biological results, and which ones are not essential \citep{linderman2017constrain,pauli2018repro}. For example, in our model, how important was it to assume that the plasticity happened only during actual collisions? Would  looming selectivity develop if instead of looming stimuli we used more general visual stimuli? Is structured sensory flow even necessary for the emergence of selectivity \citep{triplett2018emergence}? To answer these questions, we repeated our analysis several times, excluding different parts of the model one by one. First we repeated all computational experiments and analysis, but replaced STDP with simple symmetrical Hebbian plasticity. In a different set of modeling experiments, we removed explicit synaptic competition, by replacing it with synapse weight decay, and in yet another set we greatly decreased the amount of intrinsic plasticity present in the system. Finally, in last two sets of model runs, we let the model develop either while exposed to randomized oblique translational stimuli (instead of looming stimuli), or while exposed to random visual noise. The results of these 5 sensitivity analyses are presented in Figure XXX and Table 1.

We found that training specifically on looming stimuli was not critical for the model, but training on structured stimuli (as opposed to noise) was. When looming stimuli were replaced by non-collision visual stimuli (Table 1, column "Visual"), almost all network parameters still developed similarly to how they did in the base model. Selectivity to looming stimuli also developed similarly, but reached 20-50\% lower selectivity values. In contrast, when patterned stimuli were replaced with uncorrelated noise (Table 1, column "Noise"), selectivity did not improve with time, and while the network still changed its efficiency, modularity, and degree distribution, it did not acquire hierarchical structure, and positioning of selective cells within the graph remained random.

Disruption of different developmental rules led to very different changes in network development. When STDP was replaced by simple Hebbian plasticity (Table 1, column "No STDP"), looming selectivity was similar or better than with STDP, and the network generally developed similarly, except that modularity was higher (Figure 6XXX), neuronal ensembles were more strongly interconnected (Figure XXX), and reciprocal connections between pairs of neurons remained probable (Figure XXX). This is consistent with the idea that, unlike STDP, Hebbian plasticity does not eliminate tight clusters of strongly interconnected neurons \citep{fiete2010chains}. When synaptic competition was replaced with synaptic strength decay (Table 1, column "No Syn. Comp."), the degree distribution was very different (got flatter rather than sharper); selectivity for looming stimuli was disrupted; the pattern of interactions between cell selectivity and cell centrality became unlike what was observed in the base series of experiments, and the network became strongly hierarchical. The main reason for these differences appears to be that without synaptic competition, chains of connected neurons were allowed to lead to "dead-ends" within the graph, while with competition the total output of each neuron remains roughly constant, leading to the development of cycles. Finally, when intrinsic plasticity was weakened (Table 1, column "No Intrinsic"), the network did not acquire sensitivity to looming stimuli, and had simpler response structure (both in terms of PCA results, and ensemble analysis; Figure XXX), but retained most of changes in network topology. This suggests that intrinsic plasticity is critical for learning, as without it the network was primarily driven by spontaneous activity, which structured it incorrectly.

\section*{Discussion}

In the first half of this study, we described a way to reconstruct functional connectivity in the tectum of \textit{Xenopus} tadpoles from high-speed Calcium imaging recordings, and used these reconstructions to describe several novel findings about the topology of tectal network. We show that in development, tectal networks tend to become more openwork, approaching a scalefree statistics. We also show that looming-selective cells tend to occupy a special place in this networks: they tend to be located in the middle of the receptive field for a looming stimuli, and tend to serve as "information sinks", collecting more inputs from the rest of the network, compared to non-selective cells (as accessed via both in-degree statistics, and Katz centrality).

In the second part of this paper, we hypothesized that a developing network governed by spike-time-dependent plasticity, synaptic competition, and stimulated by patterned visual inputs, would 1) spontaneously acquire selectivity for looming stimuli, and 2) develop a non-random network structure. We also hypothesized that these effects would be robust enough to be replicated in biological experiments. The support for this hypothesis is mixed. The model did develop selectivity for looming stimuli (in terms of both average preference, and distributed stimulus encoding), and this improvement in selectivity was resilient to changes in developmental rules. Yet these seemingly robust results were not truly replicated in biological experiments, as we observed no improvement in looming detection over development, and neither average selectivity, stimulus encoding, nor cell specialization differed between younger and older tadpoles. This was particularly surprising in view of a known improvement in collision detection with tadpole age \citep{dong2009}.

The model also developed a non-random network structure, with a scale-free degree distribution, low clustering, and high modularity. Of these predictions, about two thirds were replicated in biological experiments (Table 1, compare columns "Imaging" and "Model / Base"): most notably, changes in network degree distribution, a disappearance of bidirectional connections, and statements related to neuronal ensembles. These matches between the model and the experiments suggest that at the very least, our model captured the nature of networks development under the influence of synaptic competition and spike-time-dependent plasticity (STDP). Synaptic competition promoted connectivity in weakly connected neurons, while "punishing" overconnected cells, which created light-frame, openwork graph structures \citep{fiete2010chains}, while STDP coordinated activity within subnetworks, increasing modularity \citep{stam2010modular,litwin2014assemblies}, similar to how it was previously described for Hebbian plasticity \citep{triplett2018emergence,damicelli2018topomod}. We did not observe changes in the \textit{number} of neuronal ensembles \citep{avitan2017spontaneous,pietri2017emergence}, but we believe that it is only because our experiments were not suited for ensemble detection, as we worked with strong shared inputs that reliably activated almost every neuron in the network. This is very different from a case of spontaneous activity, where different sub-networks get activated randomly, with activity propagating within modules more readily than between them \citep{avitan2017spontaneous}.

At the same time, most model predictions about how network properties were supposed to change in development did not replicate in biological experiments. There are four possible explanations for this discrepancy. First, while neurons in stage 46 and 49 tadpoles have different synaptic and intrinsic properties \citep{ciarleglio2015}, and while retinal inputs to the tectum continue to refine \citep{tao2005refinement, munz2014hebbian}, the patterns of internal tectal connectivity may be relatively settled by stage 46. In our model, most network measures plateaued, or even slowly reversed late in development, which means that even for a qualitative comparison between the model and the experiment we have to make a critical assumption about whether stage 46 of tadpole development corresponds to a mid-point of network maturation, or whether it falls on the developmental plateau. The absence of improvement in stimulus identity prediction from neuronal activity in older tadpoles, compared to younger ones, suggests that both stages may indeed fall on the “plateau”. If true, this would mean that less robust collision avoidance in behaving stage 46 tadpoles \citep{dong2009} is likely to be due to maturation of sensorimotor projections from the tectum to the hindbrain, which we did not assess in this study.

Second, a poor fit between model predictions and biological experiments may be a consequence of low statistical power of this study. With 14 and 16 networks for each of two developmental stages, we could only hope to detect changes of Cohen $d \approx$ 1.0, assuming $p_t<$ 0.05 threshold and 80\% power. Moreover, based on available imaging studies, we can estimate that at stage 49 tadpoles, one side of a tectum contains about 10-15 thousand neurons, as it is about 40 cells across \citep{hiramoto2009}, and packed 6-10 cells deep in its thickest part \citep{hewapathirane2008vivo}, while tapering towards the edges \citep{bollmann2009}. On the other hand, here we reconstructed connectivity within the top layer of 128$\pm$40 cells, in a field of about 12 by 12 neurons, which means that our reconstructions covered only about 1\% of a full tectal network. With a coverage so sparse, our parameter estimations were expected to be noisy, further lowering the power of our tests.

Third, one can question the validity of our connectivity reconstructions, as we did not have a technical opportunity to compare these reconstructions to any sort of "ground truth" connectivity. The best way to address this concern would be to run a set of control experiment, analyzing transfer entropy between pairs of cells proven to be either connected or disconnected, to estimate the power of graph reconstruction from Ca imaging recordings. Unfortunately, these experiments are currently beyond our technical ability, so we have to rely on indirect criteria for successful network reconstruction. Two most important observations that support the validity of our results are the fact that the share of reciprocal connections decreased in development; and that we observed a consistent non-randomness of almost all network measurements in reconstructed networks. Also, we were comforted by a good replication of tectal response shapes (compared to \citep{khakhalin2014}), a good internal replication of edge detection between stimuli types (see Methods), and an observation of retinotopy during responses to looming stimuli.

Finally, the fourth way to explain a relatively poor fit between the model and the imaging experiments is to assume that the mechanisms of looming selectivity in the tectum are different from that in the model. In the model, looming selectivity was largerly due to the development of synfire chains \citep{zheng2014synfire,cohen2002synreview} that were guided by structured sensory activation \citep{clopath2010stdpcoding}, and thus encoded a typical activation pattern in response to looming stimuli. Later, when looming stimuli were presented to a model, they "resonated" with matching synfire chains, causing stronger network activation. It maybe that in the biological tectum, enhanced responses to looming stimuli are due to either delayed recurrent activation model \citep{khakhalin2014,jang2016}, or some sort of dynamic inactivation model \citep{fotowat2011multiplexing}. Two most concerning discrepancies between the model and the experiments are  the position of selective cells within the network (central in tadpoles, peripheral in the model), and the difference in centrality measurements related to local signal integration (high in-degree and Katz rank in biological experiments, but no similar effect in the model). At the same time, the difference in selective cell position may be due to some spatial constraints that were present in a biological brain, but were not present in the model, while the difference in Katz centrality may be explained by a combination of two factors: a strict synaptic competition in the model, and incomplete observation of networks in real tecta. Indeed, in the model we forced every neuron to have outputs within the network, which promoted the development of cycles even for highly selective cells, while in the real tectum highly selective cells may lack output intratectal connections. This effect would be further exaggerated by sparse observation of intratectal connections, as highly convergent nodes would remain convergent in a sparsely observed network, while their outputs, even if they exist, would be more likely to get lost.

Despite an incomplete fit with the data, our model still yielded some interesting general predictions. The sensitivity analysis showed that of all modeling assumptions, the presence of patterned visual stimulation was most important for the development of input selectivity. The model converged to a state suitable for looming detection even if STDP, synaptic competition, or intrinsic plasticity were disrupted, but it failed if sensory inputs were kept random. Moreover, it was not critical for the model to be specifically trained on looming stimuli: it performed almost as well with a mix of oblique looming, receding, and translating stimuli, which means that it is stimulus locality and edge continuity that mattered for the development. This suggests that proper tectal network organization can emerge from responses to retinal waves alone, provided that they have spatial and temporal statistics similar to that of behaviorally relevant visual stimuli \citep{huberman2008waves}. Note that our model predicts disrupted tectal organization in enucleated or dark-reared animals, which matches experimental research in tadpoles \citep{xu2011}, but seems to contradict experiemntal observations from zebrafish \citep{pietri2017emergence}.

At the same time, in practice, to arrange a subset of tectal outputs in an actionable looming detector, as we did using multivariate logistics regression, a developing brain would need access to a learning signal. We assume that in aquatic vertebrates, this learning signal may come from both dimming receptors in the retina \citep{baranauskas2012}, and lateral line receptors in the body \citep{truszkowski2017}. These inputs may trigger plasticity tectal outputs to the reticulospinal neurons in the hindbrain, selecting a subset of inputs that are most active immediately before a collision. Moreover, during random encounters with spatial objects, different parts of the retina would be dimmed, and different segments of the lateral line would be activated in each collision, theoretically allowing the animal to build several overlapping subnetworks, selective for collisions of different geometry, and projecting to different subsets of motor neurons \citep{helmbrecht2018topography}. Therefore, this type of learning could lead to the development of spatially nuanced escape responses to optimize avoidance behaviors, as described in both tadpoles \citep{khakhalin2014} and fish \citep{bhattacharyya2017assessment}.

To sum up, we show that a combination of simple developmental rules with patterned sensory inputs can quickly shape a random network into a structured retinotopic system, able to support collision detection. Moreover, our results suggest that collision detection in small aquatic animals may partially rely on resonant subnetworks of spatially organized synfire chains that integrate information about stimulus position, speed, and size change, and transform it into an appropriate motor response. We hypothesize that this sensorimotor transformation may develop through a form of reinforcement learning in the hindbrain, and hope to explicitly test this hypothesis in the future, both experimentally, and through computational modeling.

\section*{Methods}

All code for his paper is available at: \url{https://github.com/khakhalin/Ca-Imaging-and-Model-2018}

\subsection*{Statistics and reporting}

Unless stated otherwise, all values are reported as mean $\pm$ standard deviation. For common tests, the type of a test is indicated by the subscript for its reported p-value: $p_t$ for a two-sample t-test with two tails and unequal variances; $p_{t1}$ for a one-sample two-tail t-test, and $p_r$ for a Pearson correlation test.

In this paper we describe adjacency matrices as they are used in computational neuroscience, where $w_{ji}$ is a weight of an edge coming from node $i$ to node $j$, which is different from how adjacency matrices are presented in graph theory, where $A_{ji}$ would typically mean an edge from node $j$ to node $i$ (and so $W = A^\top$).

\subsection*{Experiments}

We followed procedures previously described in \citep{xu2011,truszkowski2017}, with visual stimulation from \citep{khakhalin2014}. Experiments were performed at Brown University, in accordance with university IACUC protocols. Unless noted otherwise, chemicals were purchased from Sigma. Tadpoles were kept in Steinberg’s solution, on a 12/12 light cycle, at 18$^\circ$ C for 10-20 days, until they reached Nieuwkoop-Faber developmental stages 45-46 or 48-49. In each experiment, we anesthetized a tadpole with 0.02\% tricainemethane sulfonate (MS-222) solution for 5 minutes, then paralyzed it by immersion in 20 mM solution of tubacurarine for 5 minutes, and pinned it down to a carved Sylgard block within the recording chamber, filled with artificial cerebro-spinal fluid solution (ACSF: 115 mM NaCl, 4 mM KCl, 5 mM HEPES, 10 uM glycine, 10 mM glucose). The optic tectum was exposed, and ventricular membrane was removed on one side of the tectum. Tadpoles were pinned tilted, at an angle of 10-20$^\circ$, to keep the exposed tectal surface flat for imaging. We then surrounded the tadpole with a small circular enclosure 15 mm in diameter, made of a thicker part of a standard plastic transfer pipette, to achieve higher concentration of Ca-sensitive dye in the solution. We dissolved 50 ug of Oregon Green Bapta 1 solution (OGB1 $\#$06807, Molecular Probes, Waltham, MA) in 30 ul of medium consisting of 4\% F-127 detergent in 96\% DMSO by weight; agitated this solution in a sonicator for 15 minutes, then added 30 ul of ACSF to the vial, and sonicated for 10 minutes more. The solution was then transferred to the chamber, mixed with 4 ml of ACSF to the final concentration of 10 uM, and the chamber was placed in the dark for 1 hour. After staining, the circular enclosure was removed; the preparation was washed with 10 ml of ACSF 3 times; the chamber was filled with 10 ml of fresh ACSF, and transferred under the scope.

This staining protocol with a BAPTA-conjugated dye proved to be uniquely challenging, and had a high failure rate. As staining procedure involved a detergent, and called for high concentrations of dye, the most successful preparations were those that received the highest possible exposure that did not yet kill the cells. A large share of preparations however either fell short of optimal staining, and had a weak fluorescence signal, and low signal-to-noise ratio, or got overexposed, leading to strong fluorescence, but weak responses to stimulation, as neurons grew increasingly unhealthy. The variable amount of uncertainty in edge detection from one experiment to another also complicated our network analysis (see below).

Visual stimulation was provided with a previously described setup \citep{khakhalin2014}, consisting of an LCD screen (Kopin Corporation, Taunton, MA, USA) illuminated by a blue LED (LXHL-LB3C, 490 nm; Lumileds Lighting, USA), with the image projected to an optic multifiber (600 um, Fujikura Ltd, Tokyo, Japan). The other end of the fiber was brought to the left eye of the tadpole, and placed 400 um away from the lens, and on the axis of the eye, to have the image projected to the center of the retina. The stimulation sequence consisted of three stimuli: looming stimulus (in which a circle appeared in the center of the field, its radius growing linearly from 0 to full-field within 1 second), full-field flash, and spatially “scrambled” stimulus. For the scrambled stimulus, we divided the field of view into a grid of 17 by 17 squares and randomly reassigned these squares within the image. The result was a stimulus that was identical to looming stimulus in terms of its total brightness at every time step, and presented fragments of a moving edge locally (within every square in a reshuffling grid), but lacked mesoscale spatial organization. The permutation of squares within the grid was randomized for each experiment, but consistent within all trials within an experiment. The stimuli were delivered every 20 s, in a sequence “looming, flash, scrambled”, usually for the total of 60 or 72 stimuli. The stimuli were generated in Matlab (Mathworks), using Psychtoolbox \citep{kleiner2007psychtoolbox}. Imaging excitation light was turned on 1 s before the onset of the visual stimulus, which was shown not to interfere with visual stimulation \citep{xu2011}, and kept on for 5 s.

The tectum was imaged using a Nikon Eclipse FN1 microscope with a 60x water-immersion objective and an ANDOR 860 EM-CCD camera. NIS-elements software (Nikon) was used to record the activity, with binning of 8x8 pixels per bin, resulting in a 130x130 image covering the field of view of 1130 um. The data was acquired with 10 ms auto-exposure, which led to actual frame rate of 84 frames per second (11.9 ms per frame). For each preparation, we used a focal plane that produced images of as many cells as possible, which usually meant a plane focused “in-between” the topmost and bottom-most cells within the field of view. To keep the signal to noise ratio consistent throughout the experiment despite the ongoing bleaching of the Ca sensor, we started with relatively weak illumination (with neutral density filter ND4 engaged) and no signal amplification by the camera (EM gain of 0). We then increased the EM gain level gradually after every 12 stimuli, to keep the signal level approximately constant. Once EM gain setting reached the value of 7, we increased illumination strength by disengaging one of the density filters, reduced EM gain back to 0, and repeated the process.

Videos were processed offline; circular regions of interest of equal size (21 bins per region) were manually positioned over neurons with well defined, highly variable Ca responses. The average fluorescence within each region of interest was quantified, and exported to Matlab. We processed fluorescence traces with a non-negative deconvolution algorithm \citep{vogelstein2010oopsi}, and used its output without thresholding, interpreting it as an estimation of both timing and number of possible spikes produced by each cell. We chose this approach, as depending on the overlap each cell body had with the focal plane, and the amount of dye sequestered, different neurons had very different signal-to-noise ratios, which complicated the matter of finding a single threshold. This decision also shaped all further steps of analysis, as in our dataset both poorly resolved cells with low spiking activity were represented not by spike traces that were mostly silent, but by traces that approached a maximum entropy, uniform distribution of estimator values. For the purposes of deconvolution, in each recording the reference cell was selected automatically, as the cell with 5th highest amplitude fluorescence response.

We did not attempt to match inferred spike trains to the “ground-truth” electrophysiological recording from a reference cell, as the validity of this specific calcium imaging protocol was justified previously \citep{xu2011,truszkowski2017}. We also did not perform background subtraction \citep{truszkowski2017}, as most effects of background fluorescence were expected to be cancelled out during analysis. The main risk of not subtracting the background is that unsubtracted traces may contain a superposition of axonal spiking and synaptic activation in the neuropil. In our experiments, neuropil generally was not stained, as the dye had little physical access to the neuropil, at least compared to principal tectal neurons. Moreover, the signal acquisition was by design sensitive to fluorescence sources lying within the focal plane, which means that the neuropil signal was both greatly attenuated, and spatially averaged. Finally, average neuropil activation was expected to be similar in every trial, as same stimuli were presented to the tadpole in every trial. As deconvolution operation is close to linear, and we did not perform spike thresholding, any shared neuropil signal would be deconvolved, “hidden” in inferred spike-trains, and later cancelled out during trial-reshuffling (see below). Similarly, we did not address motion artifacts, as in our preparation they were synchronous in all cells (manifested as parallel displacement of signal sources from fixed ROIs), and therefore only introduced a fixed bias to all TE estimations.

\subsection*{Analysis}

\textbf{Basic analysis} For response amplitudes, we used average reconstructed responses between 250 and 2000 ms into the recording, as this window included full visual responses, but excluded artifacts caused by the excitation light. As a measure of stimulus selectivity for each cell, and in some cases for the selectivity of total network response, we used Cohen’s $d$ effect size for the difference between responses to looming and flash, or looming and scrambled stimuli:

\[ d = (m_L-m_F)/ \sqrt{ \big((n_L-1) s^2_L + (n_F-1) s^2_F)/(n_L + n_F - 2)} = \]
\[ =(m_L-m_F)/\sqrt{\big(s^2_L+s^2_F\big)/2} \]

in case of equal sample sizes $n_L=n_F=n$. Here $m_L$ and $m_F$ are mean responses to looming and flash stimuli respectively, and $s_L$ , $s_F$ are standard deviations for both groups.

To find the \textbf{retinotopy center}, we concatenated all responses of every cell to looming stimuli into one vector, ran a principal component analysis on these vectors, then rotated two first components using promax rotation, and made sure that the 1st component $c^1$ is the one with shorter latency, and that it is positive, flipping the components if necessary. We then ran a non-linear optimization, looking for a pair of coordinates $(x,y)$ within the field of view, that would maximize the absolute value of correlation between distances of each cell to this center and the prominence of the short-latency component in this cell:

\[ r = \text{cor}\big(\sqrt{(x_i-x)^2+(y_i-y)^2}\ ,\ c^1_i/(c^1_i + c^2_i), \big) \]

We then interpreted these coordinates as our best guess for the possible position of the "retinotopy center" for each recording. The fits were robust, with $p<$ 0.05 observed in every experiment (30/30), and average achieved correlations of $r=$ 0.59$\pm$0.23. To assess possible overfitting, we performed identical optimization fitting  after reshuffling cell identities 5 times for each experiment, which yielded average $r$ values of only 0.13$\pm$0.06, and $p_r<$ 0.05 in 21\% of experiments. From this we concluded that cells with early responses to looming stimuli were indeed clustered together, and that this clustering was not an artifact of our analysis, even though the $r$-values were probably somewhat exaggerated.

For \textbf{response latency} calculations, we looked at each response $y(t)$, and found the position of its maximum $(x_M, y_M)$. We then used the least squares fit with non-linear solved to approximate the segment between the beginning of the response and $x_M$ with a piecewise linear function:

\[ f(x) = \left \{ \begin{array}{cll} 0 & \text{for} & 0 \leqslant x<x_L \\
a (x-x_L) & \text{for} & x_L\leqslant x < x_M \end{array} \right. \]

optimizing for $x_L$ and $a$, where $x_L$ is the response latency, and $a$ is an amplitude-like parameter we did not use for subsequent analysis. This approach worked well for isolated responses with good signal to noise ratio, but got increasingly noisy with weak signals. To quantify the retinotopy, we therefore used the results of factor analysis, and only referred to response latencies for verification.

\textbf{Ensemble analysis}. To find ensembles of cells that tended to be co-active together, we used a modified spectral clustering procedure \citep{ng2002spectral} and the definitinon of spectral modularity \citep{newman2006modularity}, generalized to weighted oriented graphs. First, for each stimulus type, for each cell $i$, and separately for each experimental trial $k$, we unbiased and normalized each activity response $a^k_i(t)$, by subtracting its mean, and dividing the result over standard deviation:

\[ a^k_i(t)' = \big(a^k_i(t)-b^k_i\big)/\sigma^k_i \]

where $b^k_i = \sum_{t=1}^T{a^k_i(t)}$ and $\sigma^k_i = \frac{1}{T-1}\sum_{t=1}^T{(a^k_i(t) - b^k_i))}$ .

Then, for each cell, we calculated the average response across all trials of the same type: 

\[ \overline{a_i}(t) = \frac{1}{n}\sum_{k=1}^n{a^k_i(t)} \]

and subtracted these average responses from each trial, which resulted in a vector of a trial-by-trial deviations from the average response:

\[ a^k_i(t)'' = a^k_i(t)' - \overline{a_i}(t) \]

We then used these vectors of deviations from the mean, concatenated across all trials, to calculate a cross-correlation matrix, to see which cells tended to be unusually active or unusually inactive together:

\[ c_{ij} = \text{corr}\big(a''_i(k,t)\, , \, a''_j(k,t)\big) \]

We calculated adjusted correlations $c_{ij}$ separately for each of three types of stimuli (flash, scramble, and looming), and averaged these three estimations $c_{ij}^s$, to arrive at a, hopefully, less noisy estimation of adjusted cross-correlation. We then removed negative correlations, replacing them with zeroes.

\[ c'_{ij} = \text{max}\big(0 \, , \, \frac{1}{3} \sum_{s}{c_{ij}^s}\big) \]

We then roughly followed the spectral clustering approach by \citep{ng2002spectral}, with some adjustments that seemed appropriate for ensemble detection. We first transformed our correlation matrix $c_{ij}$ into a matrix of pairwise Euclidean distances:

\[ \varphi_{ij} = 2(1-c_{ij}) \]

and then to affinity matrix $\textbf{A}$:

\[ a_{ij} = \text{exp}(-\varphi_{ij}/\sigma) \]

where $\sigma$ is a free parameter that we set at 10000. We then calculated a diagonal degree matrix $\textbf{D}$ such that $d_{ij} = 0$ for $i \neq j$ , and $d_{ii} = \sum_k{a_ik}$ otherwise. We used $\textbf{D}$ to build a Laplacian matrix $L$, such as:

\[ L_{ij} = a_{ij}/\sqrt{d_{ii}\cdot d_{jj}} \]

and found eigenvectors $x_1$ .. $x_n$ of matrix $L$. Then we selected a number of ensembles to find $k$, going through all values from 1 (no ensembles) and up to the number of cells (each cell as a separate ensemble). For each $k$, we found first $k$ largest eigenvectors of $L$, stacked them in columns, and renormalized each row of this matrix to give it unit length:

\[ u_{lm} = \frac{x_{lm}}{\sqrt{\sum_{z=1}^{k}{x_{lz}^2}}} \]

where $x_{lm}$ is an $m$-th element of $l$-th eigenvector of $L$. We then used k-means clustering on rows of $\textbf{U}$ as points in $\mathbb{R}^k$, looking for $k$ clusters. Once rows of $\textbf{U}$ (and so cells in the original data) were assigned to $k$ clusters, we calculated spectral modularity of this partition on the original matrix $w_{ij}$, using a weighted directed modification of classic formula from \citep{newman2006modularity}:

\[ Q_k = \frac{1}{4m}\sum_{ij}{\delta_{ij}\Big(w_{ij}-\frac{d^{out}_i d^{in}_j}{2m}}\Big) \]

Here $d^{out}_i$ and $d^{in}_j$ are weighted out- and in-degrees for nodes $i$ and $j$ respectively: $d^{out}_i = \sum_k{w_{ik}}$ , and $d^{in}_j = \sum_k{w_{kj}}$ ; $m$ is the total number of edges involved: $m = \sum_{ij}{w_{ij}}/2$ , and $\delta_{ij}$ is a signal matrix with $\delta_{ij}=1$ for nodes $i$ and $j$ that belong to the same cluster, and $\delta_{ij} = 0$ otherwise. We then found the number of clusters $K$ that, after spectral clustering, produced highest modularity $Q_K$ across all $Q_k$, and used $K$ as an estimation of the number of ensembles in the network, and corresponding cluster allocation - as the allocation of cells to these ensembles.

\textbf{Network reconstruction}. For network reconstruction, we used a modified Transfer Entropy (TE) calculation, adapted from \citep{stetter2012te,gourevitch2007te}. Mathematically, fast Ca imaging recordings, as used in this study, provides a middle ground between commonly used, slower Ca imaging data and multielectrode recordings. In most Ca imaging recordings, the frame acquisition time (100 ms) is an order of magnitude longer than the transmission time between neurons ( 2 ms), which biases analysis towards co-activation analysis. In our data, the high rate of acquisition (12 ms per frame) was very close to typical cell-to-cell activation transmission time in the tectum, so we restricted our analysis to interactions between the activity of each cell at a frame i and their activity at the next frame i+1, ignoring both longer (multiframe), and same-frame interactions.

For each cell, we took its inferred activity train, and binned it at 3 levels, classifying every frame as either a frame with high, medium, or low activity. For each cell, we used 1/3 and 2/3 quantiles of its inferred activity train values as levels thresholds, so that all three types of frames were equally frequent, as this maximized information, while retaining low binning count. Then for each pair of neurons $i$ and $j$ we calculated the probability $P(k_j^1,k_j^0,k_i^0)$, which showed the conditional probability of neuron $j$ being in state $k_j^1$ (either 1, 2, or 3) at moment $t$, if this neuron was in a state $k_j^0$ at the previous frame $t-1$, and  input neuron $i$ was in state $k_i^0$ at the same frame $t-1$. From this set of probabilities, we calculated conditional probabilities of $P(k_j^1 \mid k_j^0)$, and finally calculated the total transfer entropy as

\[ T_{ij} = \sum_{lmn=1}^3{P(k_j^1=l,k_j^0=m,k_i^0=n)}\cdot \log\left(\frac{P(k_j^1=l , \mid , k_j^0=m, , k_i^0=n)}{P(k_j^1=l , \mid , k_j^0=m)}\right) \]

In our project, common drive (visual input from the retina) presents a particular problem. If detection of looming stimuli happens mainly through activation of selected synfire chains, the pattern of this activation would be necessarily synchronized with the causal transfer of excitation from one node to another. Because of that, it cannot be eliminated by methods that rely on the comparison of delays \citep{wollstadt2014te}. Instead, we eliminated the effects of common drive by reshuffling our data, and pairing activation history of each cell with activation history of other cells for reshuffled, unmatching trials recorded in response to same stimulus type. For each experiment, we calculated 1000 randomly reshuffled transfer entropy estimations, and then subtracted the average of these reshuffled TE estimations from our TE estimation, arriving at the value of adjusted TE \citep{gourevitch2007te}:

\[ T'_{ij} = T_{ij} - T^\text{shuffled}_{ij} \]

This approach is similar to the idea of analyzing subtle variations in activation from one response to another, as opposed to the analysis of activation traces themselves. As the stimuli we presented were same in every trial, the progression of the common drive over time was shared across all trials. If a connection between cells $i$ and $j$ was suggested by the analysis of reshuffled data, these cells were clearly sequentially driven by a common input, and not by a true causal connection between them.

For each TE estimation, we also calculated a corresponding p-value, to quantify whether the observed TE was significantly different from the set of TE estimations obtained on surrogate data, corresponding to a null hypothesis of no causal connections, and all network activity due to shared drive. With the computational power available, we could only afford to generate 1000 surrogate reshuffled networks for every TE calculation, which made it impossible to use the false discovery rate correction on our data, as it is recommended for large-scale studies of brain connectivity [Vicente 2011; Lindner 2011]. With $\sim$10$^2$ neurons and 10$^4$ connections the smallest possible non-zero p-value of 0.001, corresponding to finding a more extreme TE value in one out of 1000 surrogate experiments, was already larger than the Benjamini-Hochberg threshold of $\frac{k}{m}\alpha=$5e-6. With a permissive threshold of $\alpha$=0.01, each subset of data (recordings of responses to collisions, flashes, and scrambled stimuli), when taken separately, suggested the existence of 2\% to 69\% of all possible directed edges in the graph, depending on the experiment (median of 8\%). The share of edges that were independently discovered in all three types of experiments (median value of 0.1\% of all possible edges) was on average 1.8 times larger than one would expect in case of spurious and independent discovery (signrank $p$=7e-7), suggesting that the three subsets of data, originating from responses to three different stimuli, can be considered replications for the purposes of edge discovery. At the same time, when we tried to restrict our analysis only to edges that were fully replicated in all 3 sets with $\alpha < $0.05, we ended up with the median graph size of only 18 edges (6 edges in the largest connected component). The overwhelming majority (18 out of 19) of datasets with fewer than 10 reconstructed edges were recorded in the early set of experiments, when our dye transfer from DMSO to ACSF was still imperfect. As during data acquisition, we alternated between stage 46 and 49 tadpoles, the set of “weak” experiments was not biased, and consisted of 10 younger, and 9 older tadpoles. We decided to exclude these 19 “weak” experiments, and believe that restricting all analysis to remaining 30 experiments did not bias the study.

To make the edge inference more robust, for 30 experiments included in further analysis, we relaxed our criteria on edge discovery, while still giving preference to edges discovered independently in more than one subset of responses. To do so, we included in our reconstruction only edges with geometric mean of p-values below significance threshold: $\prod{p_k}<\alpha^3$ , where $p_k$ are p-values for each of three subsets of data (responses to flash, crash, and scrambled stimuli). Because of variation in staining quality and focal plane alignment, we could not use a fixed significance threshold $\alpha$ for all experiments, and instead followed an approach common in analysis of noisy networks, setting the average node degree (the ratio of network edges to network nodes, for directed graphs $=E/N$) to an arbitrary reasonable value \citep{stetter2012te}. For this study, we picked a value of 1.0 (number of edges equal to the number of nodes), which lead to 128$\pm$41 edges in each experiment on average (0.9\% of all possible edges); 50$\pm$21 weakly connected components, and 74$\pm$30 nodes in the largest connected component. The comparison of network properties (Fig) did not change qualitatively in a broad range of assumed average degrees (from $\sim$ 0.5 to 1.5), but observed effects became weaker and regressed to random effects outside of this range.

The TE approach did not distinguish between positive and negative influence of one neuron on another, so our reconstructed edges could include a mix of excitatory and inhibitory connections. To estimate the share of putative inhibitory connections, we calculated pairwise correlations between activities of individual neurons, compensating for the effect of shared inputs through trial reshuffling (similar to how it was done for TE), and looked at the sign of these correlations for pairs of neurons with TE$>$0. We found that 3$\pm$7\% of detected connections seemed inhibitory or inactivating, with no difference between developmental stages ($d$=0.55, $p_t$=0.1). According to our current understanding on the tectal architecture, deepest principal tectal neurons are not expected to be inhibitory [REF?], and the share of negative correlations tended to be lower in experiments with better signal-to-noise ratio, suggesting that at least part of observed inactivating connections may be false discoveries.

To analyze degree distributions, we calculated the sum of weights of incoming and outgoing edges, rounded them towards nearest whole number, and calculated frequencies $F_{in}(k)$ and $F_{out}(k)$ for each degree value $k$. We then fit a regression line $-\gamma k + b$ to a cloud of point $[k , log(F(k)) ]$, for in- and out-degrees separately, estimated two power constants $\gamma_{in}$ and $\gamma_{out}$, and averaged them to arrive at one balanced estimation ($\gamma$).

To quantify the share of reciprocal connections for the model data, we multiplied the weight matrix element-wise on itself transposed, and normalized this value by dividing it on the sum of squared weights: $S=\sum_{ji}{w_{ji} w_{ij}} / \sum_{ji}{w_{ji}^2}$ . This value is equal to 1 for symmetric weight matrices, 0 for matrices without reciprocal connections, and smoothly changes between these two values for "intermediate" all-positive matrices.

\textbf{Network analysis}. We reviewed lists of statistical tools applicable to weighted undirected graphs \citep{rubinov2010toolbox}, and selected a sufficiently diverse set of measures to describe  different aspects of our networks, including average connectivity, unevenness of density, and global structure. We also only included measures that are not supposed to change too strongly with the inclusion or exclusion of individual weakly connected nodes, to make sure that metrics estimations would not change catastrophically from one experiment to another because of small variations in the noise level, or a slightly more generous selection of regions of interests for calcium imaging video quantification. Examples of measures that did not satisfy this criterion are cycle order and the “small world” property, that both are sensitive to the inclusion of only a few weak long-ranged connections \citep{papo2016beware}. We used the following list of network metrics:

\textbf{Global network efficiency} was calculated using a function from the Brain Connectivity Toolbox \citep{rubinov2010toolbox} on reciprocals for graph weights $R_{ij} = 1/w_{ij}$, and was defined as:

\[ E = \frac{1}{n} \sum_{i \neq j}^n{\frac{d_{ij}}{n-1}} \]

where $d_{ij}$ is the length of the shortest path $P_{ij}$ connecting nodes $i$ and $j$: $d_{ij} = \sum_{kl \in P_{ij}}{R_{kl}}$

\textbf{Clustering coefficient} \citep{fagiolo2007} was calculated using the Brain Connectivity Toolbox, with a function that supported weighted directed graphs:

\[ C = \frac{1}{n} \sum_i{\frac {t_i}{(k^o_i+k^i_i)(k^o_i+k^i_i-1)-2\sum_j{w_{ij}w_{ji}}}} \]

where $k^o_i$ and $k^i_i$ are out- and in-degrees of node $i$ respectively, and $t_i$ is the weighted number of directed triangles that include node $i$:

\[ t_i = \sum_{j \neq i}{\sum_{k \neq i,j}{w^{1/3}_{ij}w^{1/3}_{jk}w^{1/3}_{ki}}} \]

To estimate \textbf{network modularity}, we used a function from the Brain Connectivity Toolbox, which calculated spectral modularity on a weighted directed graph \citep{reichardt2006community,leicht2008community}.

Our definition of \textbf{hierarchical flow} was inspired by \citep{mones2012hierarchy,czegel2015hierarchy}, but based on the modified Katz centrality \citep{katz1953original,fletcher2018katz}. To calculate Katz centrality, we assumed that on average, each node $j$ collected a flow of incoming signals through all edges $w_{ji}$ leading to this node. The activation arriving through edge $j\leftarrow i$ was proportional to the total activation $z_i$ of node $i$, the weight of this edge $w_{ji}$, a normalization coefficient equal to $1/\text{max}(w_{kl})$, and a damping factor of $d$=0.9. Each node also received a small amount of constant activation $(1-d)$=0.1. The total activation of each node was therefore defined as:

\[ z_j = (1-d) + \frac{d}{\text{max}_{k,l}(w_{kl})} \sum_{i \neq j}{a_i w_{ji}} \]

Each node then further redirected this activation to other nodes. This definition is very close to that of pagerank centrality \citep{page1999pagerank}, with a minor difference that the weights are not normalized to the value of total outgoing weights for each node $i$: that is, we work with raw weights of $w_{ji}$ rather than $w_{ji}/\sum_k{w_{ki}}$. It means that a node with many outputs has a strong influence over network activation, nodes with weak outgoing edges act almost as dead-ends. Similar to a standard pagerank algorithm, we found solution the stable solution of this problem iteratively, by initializing the network with equal values of centrality, and then running the equation above 100 times or until convergence. Once the distribution of centralities $z_i$ stabilized, we used the difference between the maximal Katz centrality and mean centrality across all nodes as a measure of hierarchical flow in the network \citep{mones2012hierarchy,czegel2015hierarchy}: $h = \text{max}(z_i) - \text{mean}(z_i)$.

To check whether network values described above were different from values expected on a random graph, we performed \textbf{graph randomization}, using a variant of degree-preserving reshuffling \citep{maslov2002} that we generalized for directed weighted graphs. For a network with $N_E$ edges we picked 3$\cdot N_E$ random pairs of nodes (nodes $i$, $j$, $k$, and $l$) that had strong connections from $i$ to $j$, and from $k$ to $l$, but weak connections or no connections from $i$ to $l$, and from $j$ to $k$ (we required $w_{ji}>w_{li}$ and $w_{lk}>w_{jk}$). We also required all four nodes to be different ($i \neq j \neq k \neq l$). Then we cross-wired these pairs of nodes, gradually randomizing network topology:

\[ \left \{ \begin{array}{l}  
w_{ji} \leftarrow w_{li} \\ 
w_{li} \leftarrow w_{ji} \\
w_{lk} \leftarrow w_{jk} \\
w_{jk} \leftarrow w_{lk}
\end{array} \right. \]

This approach to degree-preserving randomization is slightly different from the original formulation by \citep{maslov2002} in two ways. First, we explicitly don’t allow loops (self-edges) by requiring all four nodes be different. Second, we allow nodes $i$ and $k$, as well as $l$ and $j$ to be connected before the rewiring, and just swap corresponding edge weights, which seems to be a necessary adjustment for directed weighted graphs. It also means that, strictly speaking, for a weighted graph, our randomization only preserves out-degrees, but not in-degrees. Because of the requirement that $w_{ji}>w_{li}$ and $w_{lk}>w_{jk}$, for a binary directed graph our algorithm preserves in-degrees strictly, as it becomes identical to version by Maslov, while for nearly-binary graphs (bimodal or sparse), it tends to preserve in-degrees on average.

We also tested whether the connectivity and positioning of selective cells within the graph is in any way peculiar, by calculating Pearson correlations between cell selectivity and several different graph centrality measurements. Here, we used three centrality measures: weighted in-degree (the sum of weights of all connections to the node); Katz centrality; and clustering coefficient.

To study the distribution of cell selectivity within the graph, we used \textbf{weighted assortativity} as a measure of non-random association of selective cells into subnetworks. The formula for a mixing coefficient in a weighted directed network is given in \citep{farine2014weighted}, based on the logic from \citep{newman2003mixing} and \citep{leung2007weighted}. The original formula from \citep{newman2003mixing} for an unweighted undirected graph defines a mixing coefficient as a Pearson correlation coefficient between properties of nodes connected by edges, taken over all edges in the graph:

\[ r=\underset{ij: a_{ij}=1}{\text{corr}}(x_i,x_j) \]

leading to the following expression:

\[ r = \frac{\frac{1}{E} \sum{x_i x_j} - [\frac{1}{E} \sum{\frac{1}{2}(x_i+x_j)}]^2} {\frac{1}{E} \sum{(x_i^2+x_j^2)}-[\frac{1}{E} \sum{\frac{1}{2}(x_i+x_j)}]^2} \]

where sums are taken over all connected edges $ij: a_{ij}=1$, and $E$ is the total number of edges.

For a weighted graph an equivalent measure can be introduced by replacing summation over edges to summation over all possible pairs of nodes $ij$, with weights $w_{ij}$ introduced in each sum. The resulting expression can be rewritten in several different ways \citep{newman2003mixing,leung2007weighted,farine2014weighted,teller2014assortative}, with several alternative bulky expressions ultimately describing a weighted correlation calculated across all connected directed edges $ij$ with edge values $w_{ij}$ used as correlation weights:

\[ r=\text{cor}(x_i \, , \, x_j \, , \, w_{ij}) \]

where weighted correlation $\text{cor}(a,b,w)$ is introduced through weighted covariances: 

\[ \text{cor}(a,b,w) = \frac{\text{cov}(a,b,w)}{\sqrt{\text{cov}(a,a,w) \cdot \text{cov}(b,b,w)}} \]

that in turn are defined as: 

\[ \text{cov}(a,b,w) = \frac{\sum_i{w_i \cdot (a_i-\bar{a})(b_i-\bar{b})}}{\sum_i{w_i}} \]

with $\bar{a}$ and $\bar{b}$ representing weighted mean values: 

\[ \bar{a}=\sum_i{w_i a_i}/\sum_i{w_i} \]

Note that this definition seems to differ slightly from the one used in the Brain Connectivity Toolbox \citep{rubinov2010toolbox}.

\textbf{Unreported analyses}. For the sake of transparency, here we report the list of measures that were calculated, but were not included in the final manuscript for being superflous or confusing: four measures of weighted directed degree assortativity (in-in, in-out, out-in, and out-out); pagerank centrality; Katz centrality on reversed graphs $W^\top$; hierarchical flow for reversed graphs; node reach on direct and reversed graphs (unweighted analog of Katz centrality without attenuation); two alternative measures of cell selectivity: McFadden’s pseudo-$R$ for a logistic fit of stimulus identity to the total response of each cell, and a set of selectivity measures calculated on peak amplitudes instead of cumulative amplitudes (the results of both calculations were not qualitatively different to those reported in the paper). We also made several attempts to estimate the prevalence of directed cycles in our networks, but decided that these measures require too much validation to be included in this manuscript. For network analysis, we also attempted to compare rewired graphs to matching random Erdos graphs, but failed to build a good generalization for a case of weighted directed graphs with an adaptive edge detection threshold.

\subsection*{Developmental Model}

The model consisted of $n$=81 cells, arranged in a 9x9 grid. The model operated in discrete time, and was run for 500 epochs, 25 time steps each, or for $T$ = 12500 time steps total. Each cell was characterized by three values: its current activity $s_i(t)$ that represented its instantaneous firing rate; spiking threshold $h_i(t)$ that slowly changed over time, and a constant $\hat{s_i}$ that described the target spiking rate for each cell. The target spiking rates $\hat{r_i}$ were randomly assigned at the beginning of each simulation, and were distributed normally around 5/$n$ with a standard deviation of 1/$n$, which means that once these target spiking rates were matched, on average, at any time step, 5 out of 81 cells would be spiking. The spiking thresholds $h_i$ were initialized at the beginning of each simulation with a value $h_i(0) = 1/(n \hat{s_i}) + \mathcal{N}(0,0.1)$, where $\mathcal{N}(0,0.1)$ is a random value, normally distributed around 0 with a standard deviation of 0.1.

Cells were connected to each other with “synapses” of different strengths, represented by a weight matrix \textbf{W}, with weight $0 \leqslant w_{ji} \leqslant 1$ leading from cell $i$ to cell $j$. At the beginning of each simulation the weights were assigned random values, uniformly distributed between 0 and 1, except for self-connections (loops, $w_{ii}$) that were set to 0.

At each time step we first calculated the raw activation \textbf{A} of all neurons: $\textbf{A} = \textbf{WS} + \textbf{B}$, where \textbf{W} is the connectivity matrix, \textbf{S} is the vector of instantaneous spiking rates $s_i$ , and \textbf{B} is the sensory input (see below). For one cell, we have:

\[ a_i(t+1) = \sum_j{w_{ij}s_j(t)} + b_i(t) \]

These raw activation values were then adjusted down, by a formula representing global feedback inhibition, which helped to avoid run-away excitation early in development:

\[ a'_i(t+1) = \left \{ \begin{array}{l l} a_i(t+1)
& \text{if } \sum_j{s_j(t)} \leqslant \zeta \\ 
 & \\
a_i(t+1)\Big/ \Big(1 + \big(\sum_j{s_j(t)} - \zeta\big) \cdot \exp(- t/\tau_e)\Big) 
& \text{otherwise.} \end{array} \right. \]

Here $a'_i(t)$ is the final, adjusted value of activation for every cell; $\sum_j{s_j(t)}$ is the total sum of all cell activities at the previous time step; $\zeta$ is a constant that sets the level of total activity at which inhibition “turns on”, and that in our case was set to the size of the grid of cells \mbox{$\zeta=$ 9}. The exponent $\exp(-t/\tau_e)$ serves as an “easing” function that gradually “eases” the network from inhibition-dominated mode of operation to “free” operation, with a time constant $\tau_e=$ 0.056. This “easing” formula was a practical compromise that greatly sped up our computational experiments, as it dampened network activity early on, when network was still close to randomly connected, and so prone to seizure-like activity, but allowed the simulation run on its own later in development.

The activity of each neuron $s_i(t)$ was then calculated from its total activation $a'_i(t)$:

\[ s_i(t) = g_i\big(a'_i(t)\big) \]

using a logistic activation function: 

\[ g_i(a) = 1/\Big(1+\exp\big(c\cdot(h_i(t)-a)\big)\Big) \]

where $c$ is a steepness parameter, set at $c=$ 20, and $h_i(t)$ is the current spiking threshold of cell $i$. At the beginning of each simulation, spiking thresholds $h_i(0)$ were set to random values, uniformly distributed in a narrow band between $1/(n \hat{s_i})$ and $1/(n \hat{s_i}_i)+0.1$ . During the simulation, the thresholds $h_i(t)$ were updated at each time step, to model the effect of \textbf{intrinsic homeostatic plasticity}. For this purpose, for each cell, we kept track of its running average spiking rate $\bar{s_i}(t)$, and updated both average spiking rates and spiking thresholds $h_i(t)$ by the following formulas:

\[ \bar{s_i}(t+1) = (1-\kappa)\bar{s_i}(t) + \kappa s_i(t) \]

\[ h_i(t+1) = h_i(t) - r_h(\hat{s_i} - \bar{s_i}(t)) \]

where $\kappa=$ 0.05 is a constant that controls the rate of averaging, and $r_h=$ 0.1 is the rate at which spiking thresholds $h_i$ were allowed to adjust, to bring the discrepancy between the target spiking rate $\hat{s_i}$ and running average spiking rate $\bar{s_i}(t)$ to zero.

Once spiking of each neuron at the new time step $s_i(t)$ was calculated, we performed the \textbf{spike-time dependent plasticity} (STDP) step, and adjusted synaptic weights $w_{ji}$ linking neurons in the network. The intuition behind STDP in discrete time can be described by the following system, with options 1 and 2 not being mutually exclusive:

\[ w_{ji}(t+1) = \left \{ \begin{array}{lll} w_{ji}(t)+\epsilon, & \text{if } s_i(t)\neq 0 \text{ and } s_j(t+1)\neq 0 \\ w_{ji}(t)-\epsilon, & \text{if } s_i(t)\neq 0 \text{ and } s_j(t)\neq 0 \\ w_{ji}(t) & \text{if } s_i(t)=0\end{array} \right. \]

where $\epsilon$ is a change in synaptic weight. As in our model neuronal activity $s_i(t)$ was continuous, and we wanted the synaptic change to be proportional to the overlap in neuronal activity, the non-exclusive system above can be rewritten as:

\[ w_{ji}(t+1) = w_{ji}(t) + r_w \big(s_i(t)w_{ji}(t)s_j(t+1) - s_i(t)w_{ji}(t)s_j(t)\big) \]

or

\[ w_{ji}(t+1) = w_{ji}(t)\cdot\Big(1+r_w\big(s_j(t+1)-s_j(t)\big)s_i(t)\Big) \]

where $r_w$ is a constant that controls the level of synaptic plasticity; for this model $r_w$ = 0.25 .

Finally, we modeled \textbf{synaptic competition} by introducing a negative feedback, to limit the total sum of all inputs to each neuron, and all output of each neuron. At every time step, we used the weight matrix $\textbf{W}$ to calculate a modified matrix $\textbf{W}^\text{i}$, with sum of \textit{inputs} to each neuron normalized to a certain fixed value $g$ = 1.5, and a modified weight matrix $\textbf{W}^\text{o}$, for which the total sum of \textit{outputs} of each neuron was normalized to the same value: 

\[ w_{ji}^\text{i} = g \cdot w_{ji}/\sum_k{w_{jk}} \]
\[ w_{ji}^\text{o} = g \cdot w_{ji}/\sum_k{w_{ki}} \]

We then “moved” our actual weight matrix in the direction of the average of these two normalized matrices:

\[ w_{ji}(t+1) = 0.4 \cdot w_{ji} + 0.3 \cdot w^\text{i}_{ji} + 0.3 \cdot w^\text{o}_{ji} \]

Networks were activated with \textbf{simulated visual stimuli} that resembled sensory activation a real animal would have experienced when navigating in a bright-lit environment with sparsely placed black spheres. For general visual stimulation (only used in sensitivity experiments), we repeatedly created unique collision events, with randomized original position of a black sphere relative to the tadpole, final distance to the tadpole, and direction of movement through the visual field. We would then move the projection of this virtual sphere across the virtual retina over a course of $\tau=$ 10 time frames.  When training on looming stimuli (main series of computational experiments), we still initiated objects at random points within the visual field, but made sure that they approached the eye on a "collision trajectory", and at some point covered the entirety of the visual field. When training on noise, we generated random noise with $\zeta=$ 9 pixels flicking on at any given time. For looming and "general visual" stimuli, a projection of a sphere on the virtual retina was a solid circle, with its center moving linearly $(x,y) = (x_0,y_0)+(v_x,v_y)\cdot t/\tau$, and circle radius changing as $R(t) = R_0/(d_0 - v_z \cdot t/\tau)$. The virtual retina consisted of 81 pixels, arranged in a 9x9 grid, that generated both "ON" and "OFF" responses without delay or bursting, as an exclusive OR of two consecutive projections $\text{in}(t) = \text{XOR}(\text{img}(t),\text{img}(t-1))$, and had one-to-one projections to nodes in the model network.For testing, we compared responses to flashes, crashes, and scrambled stimuli. "Crashes" were different from looming stimuli in that the change of projection radius with time was linear $R(t) = \zeta/2 \cdot \sqrt{2} \cdot t/\tau$, rather than realistic; this was because we used linearly expending looming stimuli in biological experiments, both in this study, and in earlier studies \citep{khakhalin2014}. "Scrambled" stimuli were identical to "Crashes", but with all 81 pixels randomly reassigned. "Flashes" were modeled as very fast looming stimuli that took exactly 2 frames to fill the entire field of view, and with pixels reshuffled; as our model was deterministic, we had to use this approximation to introduce some variability into responses to flashes, while keeping them as close to instantaneous as possible.

While testing networks trained on different sensory stimuli, we ran into a surprising complication: during training, different stimuli provided different levels of average activation, and so not only differently shaped synaptic connections between different cells, but also, because of intrinsic plasticity, resulted in neurons acquiring different activation thresholds. This difference in excitability however was an artifact of our training method, and did not approximate any real biological phenomena, as in real tadpoles visual stimuli are expected to be relatively sparse, while we fed all stimuli to the network as one intense train with no gaps. We therefore decided to let all spiking thresholds settle down before testing, to a state that was dependent only on synaptic connectivity, and not on recent stimulation history. We let the model develop for 2000 additional time steps, with only homeostatic plasticity rule on, but without STDP or synaptic scaling, while feeding all neurons with Poisson random noise that activated on average $\zeta$ = 9 neurons at each time step. 

The effect of this additional calibration step was so prominent in the model, that we hypothesize that it may be indirectly relevant to the biological tectum as well. To maintain the network of synaptic connections, each ensemble of synfire chains has to be regularly activated, yet the more active it is, the less excitable the neurons become, making it less likely to "win" during competition with other ensembles during stimulus detection. The dynamics of plasticity in the brain would therefore pose a meta-balancing problem \citep{zenke2017temporal}: if intrinsic plasticity is too flexible, the network that detects unusual stimuli, will get spontaneously activated in the absence of these stimuli, producing high false-positive rate, and will quickly habituate to actual stimuli, but will have no trouble maintaining synaptic connections required for stimulus detection \citep{litwin2014assemblies}. If however intrinsic plasticity is too slow, the network may find it easier to maintain “optimal” levels of sensitivity, but may have trouble maintaining synaptic connections between stimulus presentations \citep{triplett2018emergence}, as low sensitivity would mean low rate of spontaneous replay. The potential solution to this problem may involve either transitioning through distinct developmental stages with different levels of intrinsic plasticity (similar to how we did it in the model, and reminiscent of a known spike of excitability in stage 47 tadpoles \citep{hamodi2014,ciarleglio2015}), or distinct maintenance and operation physiological states regulated by modulatory inputs to the network; a description that resembled that for different phases of sleep.

For model \textbf{sensitivity analysis}, we removed or attenuated parts of the model, one part at a time (but not cumulatively). We tried the following combinations:

\textbf{Non-looming stimuli}. In this mode, instead of training the model exclusively on looming stimuli, we let black circles traverse the visual field in random directions, while either staying of constant angular size, increasing in size, or decreasing in size. As a result, this type of stimulation was still spatially patterned, but consisted mostly of translational stimuli, with some oblique looming stimuli and oblique receding stimuli.

\textbf{Random stimulation}. The network was stimulated with random noise. Each “pixels” of the image would fire with the same probability of $1/\zeta$ where $\zeta$=9.

\textbf{No STDP plasticity}. Instead of a described above equation 

\[ w_{ji}(t+1) = w_{ji}(t)\cdot\Big(1+r_w\big(s_j(t+1)-s_j(t)\big)s_i(t)\Big) \]

we used an equation with symmetrical Hebb plasticity, and no negative depression term: 

\[ w_{ji}(t+1) = w_{ji}(t)\cdot\Big(1+r_w\big(s_j(t+1)\big)s_i(t)\Big) \]

\textbf{No synaptic competition}. Instead of sliding renormalization of all inputs and outputs of each neuron, we allowed synaptic weights to decay to zero: $w_{ji}(t+1) = w_{ji}(t)\cdot (1-\beta)$, where $\beta$ = 0.001.

\textbf{Weak homeostatic plasticity}. In the formula for homeostatic plasticity, instead of change coefficient $r_h$=0.1 we used $r_h$=0.01.

\section*{Acknowledgements}

My greatest gratitude is to Carlos Aizenman who encouraged me to try to publish this work as a single author, even though all experiments described here were performed on his equipment, and the materials were paid for by the money from his grant (NSF IOS-1353044). I also thank Heng Xu (Shanghai Jiao Tong University) for his help with first imaging experiments; Joshua Vogelstein (John Hopkins) for his advice on adaptive thresholding; Petko Bogdanov (SUNY Albany), Csilla Szabo (Skidmore College), Gerrit Ansmann (Bonn University), and Jim Belk (St Andrews University) for their help with network science and graph theory, and Sven Anderson (Bard College) for advice on model analysis.

% \section*{Declaration of Interests}

The authors have no conflicts of interest to disclose.

\nolinenumbers
\bibliographystyle{apalike} % For author-year
%\bibliographystyle{unsrtnat} % For Nature-style
\bibliography{refs}

%TC:endignore
\end{document}